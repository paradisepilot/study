
\section{The August 2013 Bitcoin Thefts}
\setcounter{theorem}{0}
\setcounter{equation}{0}

%%%%%%%%%%%%%%%%%%%%%%%%%%%%%%%%%%%%%%%%

The attack underlying the Bitcoin thefts of August 2013 was based
on the following (see \cite{Schneider20130128}):
\begin{proposition}
\mbox{}
\vskip 0.1cm
\noindent
Suppose we are given an ECDSA (Elliptic Curve Digital Signature Algorithm) system:
\begin{itemize}
\item
	$(E,\mathcal{O})$ is an elliptic curve over $\Z_{p}$, for some large prime $p$,
	and $\mathcal{O} \in E$.
\item
	$G \in E$ is an element of large prime order $q \in \N$.
\item
	$H : \Sigma^{*} \longrightarrow \Sigma^{n} \cong \{0,1,2,\ldots,N\}$\,
	is a cryptographic hash function,
	\,$\Sigma$\, is its alphabet, and \,$N \geq q$.
\end{itemize}
Then, given two signed messages
\begin{equation*}
\left(\,m_{1}\,\overset{{\color{white}.}}{;}\,(r,s_{1})\,\right)
\,,\,
\left(\,m_{2}\,\overset{{\color{white}.}}{;}\,(r,s_{2})\,\right)
\;\; \in \;\;
	\Sigma^{*} \times \Z_{q}^{*} \times \Z_{q}^{*}
\end{equation*}
signed using the same (static private) signing key $a \in \Z_{q}^{*}$ and
the same (``ephemeral'' private) key $k \in \Z_{q}^{*}$,
the private signing key $a \in \Z_{q}^{*}$ can be computed as follows:
\begin{equation*}
a
\;\;=\;\;
	- \left(\,H(m_{1}) \cdot s_{2} \, \overset{{\color{white}1}}{-} \, H(m_{2}) \cdot s_{1} \right)
	\cdot
	\left(\,r \overset{{\color{white}1}}{\cdot} \left(\, s_{2} \,-\, s_{1} \,\right)\right)^{-1}
	 \mod q
\end{equation*}
\end{proposition}

\proof

\vskip 0.3cm
\noindent
Recall that, by definition, we have:
\begin{equation*}
s_{1} \;=\; k^{-1} \cdot \left(\,H(m_{1}) + a \cdot r \,\right) \!\!\!\mod q\,,
\quad\quad\quad\quad
s_{2} \;=\; k^{-1} \cdot \left(\,H(m_{2}) + a \cdot r \,\right) \!\!\!\mod q\,,
\end{equation*}
where \,$r \,:=\, x_{R} \!\mod q$\, and \,$k \cdot G = (\,x_{R}\,,\,y_{R}\,) \in \Z_{p} \times \Z_{p}$.\,
Rearranging yields
\begin{equation*}
k \cdot s_{1} \;=\; H(m_{1}) \,+\, a \cdot r \!\mod q\,,
\quad\quad\quad\quad
k \cdot s_{2} \;=\; H(m_{2}) \,+\, a \cdot r \!\mod q
\end{equation*}
Multiplying both sides of the first equation with $s_{2}$ and
both sides of the second with $s_{1}$ yields:
\begin{equation*}
k \cdot s_{1} \cdot s_{2} \;=\; H(m_{1}) \cdot s_{2} \,+\, a \cdot r \cdot s_{2} \!\mod q\,,
\quad\quad\quad\quad
k \cdot s_{2} \cdot s_{1} \;=\; H(m_{2}) \cdot s_{1} \,+\, a \cdot r \cdot s_{1} \!\mod q
\end{equation*}
Subtracting the second equation above from the first yields:
\begin{equation*}
- \, H(m_{1}) \cdot s_{2} \, + \, H(m_{2}) \cdot s_{1}
\;\;=\;\;
	a \cdot r \cdot s_{2} \,-\, a \cdot r \cdot s_{1}
\;\;=\;\;
	a \cdot r \cdot \left(\, s_{2} \,-\, s_{1} \,\right) \mod q
\end{equation*}
Isolating $a$ now yields:
\begin{equation*}
a
\;\;=\;\;
	- \left(\,H(m_{1}) \cdot s_{2} \, \overset{{\color{white}1}}{-} \, H(m_{2}) \cdot s_{1} \right)
	\cdot
	\left(\,r \overset{{\color{white}1}}{\cdot} \left(\, s_{2} \,-\, s_{1} \,\right)\right)^{-1}
	 \mod q\,,
\end{equation*}
as desired.
\qed

%%%%%%%%%%%%%%%%%%%%%%%%%%%%%%%%%%%%%%%%
\begin{remark}
\mbox{}
\vskip 0.1cm
\noindent
By the above Proposition, an attacker will be able to recover the secret
private signing key \,$a \in \Z_{q}^{*}$\, of a Bitcoin address, based only on
publicly viewable information, whenever the attacker can find two
Bitcoin transactions (signed messages):
\begin{equation*}
\left(\,m_{1}\,\overset{{\color{white}.}}{;}\,(r_{1},s_{1})\,\right)
\,,\,
\left(\,m_{2}\,\overset{{\color{white}.}}{;}\,(r_{2},s_{2})\,\right)
\;\; \in \;\;
	\Sigma^{*} \times \Z_{q}^{*} \times \Z_{q}^{*}
\end{equation*}
of the same Bitcoin address satisfying:
\begin{itemize}
\item
	the two signed messages were generated
	with the same (private) ephemeral key $k \in \Z_{q}^{*}$\,,
	%which will necessarily imply \,$r_{1} \,=\, r_{2}$\,,
	and
\item
	the two signed messages also satisfy:
	\begin{equation*}
	s_{1} \,\neq\, s_{2}
	\end{equation*}
\end{itemize}
Recall also that if the two signed messages were generated with the
same ephemeral key \,$k \,\in\, \Z_{q}^{*}$\,, then we must have
\,$r_{1} \,=\, r_{2}$.\,
Therefore, operationally speaking, the attacker can proceed by
searching for a pair of signed messages (Bitcoin transactions)
whose $r$-values are equal and whose $s$-values are distinct,
for each given Bitcoin address.
Once such a pair of Bitcoin transactions have been identified
for a given Bitcoin address, the attacker can involve the
Proposition to determine the private signing key of that Bitcoin
address, and then proceed to steal the Bitcoins assigned to
that address (e.g. by transferring those Bitcoins to the attacker's
own Bitcoin address).
\end{remark}

%%%%%%%%%%%%%%%%%%%%%%%%%%%%%%%%%%%%%%%%
\begin{remark}
\mbox{}
\vskip 0.1cm
\noindent
Bitcoin thefts exploiting the above Proposition have indeed occurred.
They were apparently first reported on Saturday, August 10, 2013
(see \cite{BitcoinTalk20130810}).
The thefts concerned certain Android Bitcoin wallets.
Updates of several Bitcoin wallet Android apps were available
as early as the following day \cite{BitcoinOrg20130811}.

The underlying vulnerability was caused by the fact that
when instances of the Java classes
\emph{\texttt{SecureRandom}}, \emph{\texttt{KeyGenerator}},
\emph{\texttt{KeyPairGenerator}}, \emph{\texttt{KeyAgreement}}, and
\emph{\texttt{Signature}} in Android's Java Cryptography Architecture (JCA)
invoked the system-provided OpenSSL PRNG
(pseudo-random number generator),
%in order to generate pseudo-random numbers,
(the instances of) these Java classes failed to properly initialize
the Open SSL PRNG. See \cite{AndroidDev20130814}.

In the cases of the Bitcoin thefts, certain Android Bitcoin wallets used
Android's JCA for ephemeral key generation (during execution of ECDSA),
and the aforementioned vulnerability led to the unintentional reuse of some
of these ephemeral keys, which were exploited by attackers.

\vskip 0.3cm
\noindent
Mitigations:
\begin{itemize}
\item
	Android Bitcoin wallet developers: Update their apps to enforce key
	rotation ASAP.
	\begin{itemize}
	\item
		Make the supposedly ephemeral keys truly ephemeral.
	\item
		Seed OpenSSL PRNG with values from
		\emph{\texttt{/dev/random}} or \emph{\texttt{/dev/urandom}}.
	\end{itemize}
\item
	Android Bitcoin wallet users:
	Stop using the compromized Bitcoin addresses.
	\begin{itemize}
	\item
		Create new and secure Bitcoin addresses with updated apps.
	\item
		Transfer all Bitcoins in compromized addresses to new and secure ones.
	\end{itemize}
\end{itemize}
\end{remark}

%%%%%%%%%%%%%%%%%%%%%%%%%%%%%%%%%%%%%%%%
