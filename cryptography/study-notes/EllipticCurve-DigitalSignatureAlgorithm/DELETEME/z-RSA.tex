
\section{The RSA Public Key Cryptosystem}
\setcounter{theorem}{0}
\setcounter{equation}{0}

%%%%%%%%%%%%%%%%%%%%%%%%%%%%%%%%%%%%%%%%%%%%%%%%%%%%%%%%%%%%%%%%%%%%%%%%%%%%%%%%%%%%%%%%%%%%%%%%%%%%
\subsection{Mechanism}

\textbf{Goal:} Alice wants to send Bob an encrypted message through an insecure channel.
\begin{itemize}
\item	Bob chooses his public key $(n,e) \in \N^{2}$ and private key $d \in \N$.
	Bob publishes his public key.
	\begin{itemize}
	\item	$n \in \N$ is called the modulus, with $n = pq$, where $p$ and $q$ are large distinct
		prime numbers.  Note that Bob publishes $n$ but keeps $p$ and $q$ secret.
	\item	$e \in \N$ is the called encryption exponent, and satisfies $\gcd(e,(p-1)(q-1)) = 1$.
	\item	$d \in \N$ is the called decryption exponent, and is determined by $e$ and $n = pq$
		via $d = e^{-1} \in \Z_{(p-1)(q-1)}$.
		Note that $e^{-1} \in \Z_{(p-1)(q-1)}$ exists since $\gcd(e,(p-1)(q-1)) = 1$.
	\end{itemize}
\item	Alice
	\begin{itemize}
	\item	chooses plaintext $m \in \Z_{n} = \Z/n\Z$.
	\item	encrypts her plaintext $m$ using Bob's public key $(n,e)$
		by {\color{red}raising $m \in \Z_{n}$ to the $e^{\textnormal{th}}$ power}.
		In other words, Alice computes her ciphertext $c = m^{e} \in \Z_{n}$.
	\item	sends to Bob through the insecure channel the ciphertext $c \in \Z_{n}$.
	\end{itemize}
\item	Bob decrypts the ciphertext $c \in \Z_{n}$ from Alice by
	{\color{red}taking the $e^{\textnormal{th}}$ root}
	of $c$ in $\Z_{n}$ using his private key $d \in \N$ as follows:
	\begin{equation*}
	c^{d} = \left(m^{e}\right)^{d} = m^{ed} = m^{1+k(p-1)(q-1)} = m \cdot (m^{(p-1)(q-1)})^{k}
	= m \cdot (1)^{k} = m \in \Z_{n}
	\end{equation*}
	\begin{itemize}
	\item	The second last equality follows from $m^{(p-1)(q-1)} \equiv 1 \mod n$, which 
		follows immediately from Euler's Theorem.  It can also be justified with
		Fermat's Little Theorem as follows:
		\begin{equation*}
		\textnormal{Fermat's Little Theorem}
		\;\;\Longrightarrow\;\;
		\left\{\begin{array}{l}
		m^{(p-1)(q-1)} = \left(m^{p-1}\right)^{q-1} \equiv 1 \mod p, \;\;\textnormal{and} \\
		m^{(p-1)(q-1)} = \left(m^{q-1}\right)^{p-1} \equiv 1 \mod q.
		\end{array}\right.
		\end{equation*}
		Hence, $m^{(p-1)(q-1)} - 1$ is divisble by both $p$ and $q$, and hence also by
		$pq = n$ (since $p$ and $q$ are distinct primes).
		Thus, $m^{(p-1)(q-1)} \equiv 1 \mod n = pq$.
	\end{itemize}
\end{itemize}

%%%%%%%%%%%%%%%%%%%%%%%%%%%%%%%%%%%%%%%%%%%%%%%%%%%%%%%%%%%%%%%%%%%%%%%%%%%%%%%%%%%%%%%%%%%%%%%%%%%%
\subsection{Comments}

\begin{itemize}
\item	One-way function (easy): Exponentiation in $\Z_{n}$.
	\begin{itemize}
	\item	Repeating Squaring Algorithm
	\end{itemize}
\item	(Difficult) inverse function: Taking roots in $\Z_{n}$, for $n = pq$, where $p$ and $q$
	are large distinct prime numbers.
\item	Trapdoor: If the factorization of $n = pq$ is known, then we can convert the inverse
	function (taking roots in $\Z_{n}$, which is slow) to an exponentiation in $\Z_{n}$,
	which is fast.
\end{itemize}

%%%%%%%%%%%%%%%%%%%%%%%%%%%%%%%%%%%%%%%%%%%%%%%%%%%%%%%%%%%%%%%%%%%%%%%%%%%%%%%%%%%%%%%%%%%%%%%%%%%%
\subsection{How to find large prime numbers?}

\begin{itemize}
\item	Generate a large $N$-bit (say $N = 1024$) random number $x$, i.e. $2^{N-1} < x < 2^{N}$.
	Use an efficient primality test to check whether $x$ is prime.
	If so, we are done.  If not, repeat until we succeed.
\item	The Prime Number Theorem (from Analytic Number Theory) gives an estimate of how many times
	we need to try before succeeding.
	The Prime Number Theorem states that
	\begin{equation*}
	\lim_{x \rightarrow \infty} \dfrac{\pi(x)/x}{1/\ln(x)} \; = \; 1,
	\end{equation*}
	where $\pi(x)$ is the number of prime numbers less than or equal to $x$.
	Hence, it implies that, for large values of $N$,
	the probability that a randomly selected integer $x \in (2^{N-1},2^{N})$ is prime is
	approximately
	\begin{equation*}
	\dfrac{1}{\ln(2^{N})}
	\end{equation*}
	Conversely, this implies that, on average, out of every
	$\ln(2^N) = N\cdot\ln(2) \approx 0.693\cdot N$ randomly and independently selected
	integers from $(2^{N-1},2^{N})$, one of them will be a prime number.
	For example, if $N = 1024$, then $0.693\cdot N \approx 709.78$; in other words,
	if we are selecting random integers from $(2^{1023},2^{1024})$, then on average, we
	expect repeating approximately 710 times before we succeed in selecting a prime number.
	Note that $2^{1023} = 10^{1023\times\log_{10}(2)} \approx 10^{1023\times0.301} \approx 10^{307.95}$.
\item	The Miller-Rabin Primality test
	\begin{itemize}
	\item	\textbf{Proposition}\quad
			Let $p$ be an odd prime and write $p-1 = 2^{k}t$, where $t$ is odd.
			Then, for each $a \in \Z$ with $p \nmid a$, one of the following is true:
			\begin{itemize}
			\item[$\bullet$]	$a^{t} \equiv 1 \mod p$,\; or
			\item[$\bullet$]	One of $a^{t}$, $a^{2t}$, $a^{4t}$, $\ldots$, $a^{2^{k-1}t}$ is congruent to $-1\mod p$.
			\end{itemize}
	\item	\textbf{Corollary}\quad
			Let $n \in \Z$ be an odd number, with $n-1 = 2^{k}t$, $t$ being odd.
			Then, $n$ is composite, if any of the following is true:
			\begin{itemize}
			\item[$\bullet$]	There exists $a \in \Z$ such that $\gcd(a,n) > 1$.
			\item[$\bullet$]	There exists $a \in \Z$ such that $\gcd(a,n) = 1$, and $a^{t} \not\equiv 1 \mod n$,
							and $a^{2^{i}t} \not\equiv -1 \mod n$, for each $i = 0, 1, 2, \ldots, k-1$.
			\end{itemize}	
	\item	\textbf{Proposition}\quad
			Let $n$ be an odd composite number.
			Then, at least $75\%$ of integers between $1$ and $n-1$ are Miller-Rabin witnesses for $n$.
	\end{itemize}
\end{itemize}

%%%%%%%%%%%%%%%%%%%%%%%%%%%%%%%%%%%%%%%%%%%%%%%%%%%%%%%%%%%%%%%%%%%%%%%%%%%%%%%%%%%%%%%%%%%%%%%%%%%%
\subsection{Factorization algorithms}

\begin{itemize}
\item	Pollard's $p-1$ factorization algorithm
		\vskip 0.1cm
		This method {\color{red}``probably"} works for producing a non-trivial factor for composite $n \in \N$
		admitting a prime factor $p$ such that {\color{red}$p-1$ is a product of small primes.}
		\begin{itemize}
		\item	\textbf{Proposition:}\quad
				Let $n = pq$, where $p$ and $q$ are distinct prime numbers.
				Then the following two statements hold:
				\begin{itemize}
				\item[$\bullet$]	For each $L \in \N$ and $a$ with $p \nmid a$, we have the following implication:
								\begin{equation*}
								(p-1) \mid L
								\quad \Longrightarrow \quad L = (p-1)K
								\quad \Longrightarrow \quad a^{L}  = \left(a^{(p-1)}\right)^{K} \equiv 1 \;\textnormal{mod}\; p
								\quad \Longrightarrow \quad p \mid (a^{L} - 1)
								\end{equation*}				
				\item[$\bullet$]	For any $a \in \N$ with $p \nmid a$ and $q \nmid a$, and any $L \in \N$, 
								\begin{equation*}
								\left.\begin{array}{l}
								p \mid (a^{L} - 1) \\
								q \nmid (a^{L} -1) \\
								\end{array}\right\}
								\quad\Longrightarrow\quad
								p = \gcd(a^{L}-1,n)
								\end{equation*}
				\end{itemize}
		\item	\textbf{Key observations:}
				\begin{itemize}
				\item[$\bullet$]	If $p-1$ is a product of small primes, then $(p-1) \mid N!$, for some not-too-large $N$.  
				\item[$\bullet$]	If $(q-1) \nmid N!$, then $q \nmid (a^{N!}-1)$ is ``probably" true.
				\item[$\bullet$]	If $p-1$ is a product of small primes, and $q-1$ is NOT so, then
								computing $\gcd(a^{k!}-1,n)$, for $k = 2,3,\ldots$, will ``probably" yield
								$p$ as a non-trivial factor of $n$.
				\end{itemize}
		\end{itemize}
\item	Factorization via difference of squares
		\begin{itemize}
		\item	\textbf{Key observations:}\vskip 0.1cm
				Suppose we know that $n \in \N$ is odd and composite.
				We want to find a non-trivial factor of $n$.
				\begin{itemize}
				\item[$\bullet$]	If we can find $a, b\in\N$ such that $n$ is the difference of their squares,
								i.e. $n = a^{2} - b^{2} = (a-b)(a+b)$,
								then computing $\gcd(a-b,n)$ will yield a non-trivial factor of $n$.
				\item[$\bullet$]	Conversely, suppose $n = cd$.  Since $n$ is odd, both $c$ and $d$
								must also be odd.
								Hence, $a := \frac{1}{2}(c+d) \in \Z$ and $b := \frac{1}{2}(c-d) \in \Z$.
								And, $a^{2} - b^{2} = \cdots = cd = n$.
								In other words, every composite odd integer can be written as the
								difference of two squares.
				\item[$\bullet$]	If some multiple $kn$ is a difference of squares, i.e.
								$kn = a^{2} - b^{2} = (a-b)(a+b)$,
								then computing $\gcd(a-b,n)$ will ``probably" yield a non-trivial factor
								of $n$, since it should be unlikely that $n$ divides $a-b$.
				\item[$\bullet$]	{\color{red}In summary, if we could find $a,b\in\Z$ such that
								$a^{2} \equiv b^{2} \mod n$, then computing $\gcd(a-b,n)$ will
								probably yield a non-trivial factor of $n$.}
				\end{itemize}
		\item	\textbf{Outline of general procedure:}\vskip 0.1cm
				\begin{enumerate}
				\item	\textbf{Find $B$-smooth perfect squares in $\Z_{n}$.}
						Find many $a_{1}, a_{2}, \ldots, a_{r} \in \Z$ such that
						every prime factor of $c_{i} \equiv a_{i}^{2}\mod n$ is less than or equal to $B$.
				\item	Find sub-collections $c_{i_{1}}, c_{i_{2}}, \ldots, c_{i_{s}}$ such that
						$c_{i_{1}} c_{i_{2}} \cdots c_{i_{s}} \equiv b^{2} \mod n$ are
						perfect squares in $\Z_{n}$.
				\item	Let $a := a_{i_{1}} a_{i_{2}} \cdots a_{i_{s}} \mod n$.
						Then, computing $\gcd(a-b,n)$ will probably yield a non-trivial factor of $n$.
				\end{enumerate}
		\item	Comments on the general procedure:
				\begin{itemize}
				\item[$\bullet$]	Step (3) can be performed efficiently using the Euclidean Algorithm.
				\item[$\bullet$]	Step (2) is equivalent to solving a homogeneous (sparse) system of
								linear equations over $\F_{2}$.
				\item[$\bullet$]	The main challenge in difference-of-squares factorization is Step (1),
								namely, given $n \in \Z$, finding enough $B$-smooth perfect squares
								in $\Z_{n}$.
				\end{itemize}
		\end{itemize}
\end{itemize}
