
\section{Block Ciphers}
\setcounter{theorem}{0}
\setcounter{equation}{0}

%%%%%%%%%%%%%%%%%%%%%%%%%%%%%%%%%%%%%%%%
\begin{itemize}
\item
	Core encryption map:
	\begin{equation*}
	E : \Sigma^{n} \times \mathcal{K} \longrightarrow \Sigma^{n}\,
	\end{equation*}
	\begin{itemize}
	\item
		$\Sigma$ is a finite set, and serves as the alphabet of the plain texts as well as cipher texts.
	\item
		$n \in \N$ is the block size
	\item
		$\mathcal{K}$ is a finite set, and serves as the key space.
	\item
		For each key $k \in \mathcal{K}$, the map
		\begin{equation*}
		E(\,\cdot\,;k) : \Sigma^{n} \longrightarrow \Sigma^{n}
		\end{equation*}
		is a bijection.
		Each $E(\,\cdot\,,k)$ bijectively maps strings of characters in $\Sigma$ of length $n \in \N$
		to strings of characters in $\Sigma$ of length $n$.
	\end{itemize}
\item
	Special cases (some simple classical -- insecure -- cryptosystems):
	\begin{itemize}
	\item
		\textbf{Substitution ciphers}: block ciphers of block size $n = 1$.
		Thus, a substitution cipher encrypts by permuting the labels of the characters in the alphabet $\Sigma$.
	\item
		\textbf{Permutation ciphers}: $\mathcal{K} = S_{n}$,
		\begin{eqnarray*}
		E\!\left(\,x_{1}x_{2} \cdots x_{n}\,;\pi\,\right)
		& = &
			\;\; x_{\pi(1)} \;\;\, x_{\pi(2)} \;\;\, \cdots \;\; x_{\pi(n)} \,,
		\quad\;\,
		\textnormal{for each \,$\pi \in S_{n}$\, and \,$x \in \Sigma^{n}$}
		\\
		D\!\left(\;y_{1}y_{2} \cdots y_{n}\;;\pi\,\right)
		& = &
			y_{\pi^{-1}(1)}\,y_{\pi^{-1}(2)} \, \cdots \, y_{\pi^{-1}(n)}\,,
		\quad
		\textnormal{for each \,$\pi \in S_{n}$\, and \,$y \in \Sigma^{n}$}
		\end{eqnarray*}
	\item
		\textbf{Affine block ciphers}: $\Sigma = \Z_{m} = \Z / m\Z$, for some $m \in \N$,
		\begin{equation*}
		\mathcal{K}
		\; = \;
			\left\{\,
			\left.
				(A,b) \in \Z_{m}^{n \times n} \overset{{\color{white}.}}{\times} \Z_{m}^{n}
			\;\,\right\vert\;
				\det(A) \in \Z_{m}^{\times}
			\;\right\}
		\; = \;
			\left\{\,
			\left.
				(A,b) \in \Z_{m}^{n \times n} \overset{{\color{white}.}}{\times} \Z_{m}^{n}
			\;\,\right\vert\;
				\gcd\!\left(\det(A),\overset{{\color{white}.}}{m}\right) = 1
			\;\right\}
		\end{equation*}
		\begin{eqnarray*}
		E\!\left(\,x\,;A,b\,\right)
		& = &
			A \cdot x + b\,,
		\quad\quad\;\;\,
		\textnormal{for each \,$(A,b) \in \mathcal{K}$\, and \,$x \in \Z_{m}^{n}$}
		\\
		D\!\left(\,y\,;A,b\,\right)
		& = &
			A^{-1} \cdot (y - b)\,,
		\quad
		\textnormal{for each \,$(A,b) \in \mathcal{K}$\, and \,$y \in \Z_{m}^{n}$} 
		\end{eqnarray*}
	\item
		\textbf{Vigen\`ere cipher}: affine block cipher with $A = I_{n \times n}$,
		hence $\mathcal{K} = \Z_{m}^{n}$.
	\item
		\textbf{Hill cipher (linear block cipher)}: affine block cipher with $b = 0_{n} \in \Z_{m}^{n}$,
		hence
		\begin{equation*}
		\mathcal{K}
		\; = \;
			\left\{\,
			\left.
				A \overset{{\color{white}.}}{\in} \Z_{m}^{n \times n}
			\;\,\right\vert\;
				\det(A) \in \Z_{m}^{\times}
			\;\right\}
		\; = \;
			\left\{\,
			\left.
				A \overset{{\color{white}.}}{\in} \Z_{m}^{n \times n}
			\;\,\right\vert\;
				\gcd\!\left(\det(A),\overset{{\color{white}.}}{m}\right) = 1
			\;\right\}
		\end{equation*}
	\item
		\textbf{Affine permutation cipher}: a Hill cipher (hence, also an affine block cipher)
		where each $A \in \Z_{m}^{n \times n}$ is a permutation matrix.
		(Note that an affine permutation cipher is equivalent to a permutation cipher
		whose alphabet is $\Z_{m}$.)
	\end{itemize}
\item
	Modes of operation:
\end{itemize}

%%%%%%%%%%%%%%%%%%%%%%%%%%%%%%%%%%%%%%%%
