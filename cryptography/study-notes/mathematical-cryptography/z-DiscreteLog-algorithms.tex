
\section{Algorithms for solving the Discrete Logarithm Problem}
\setcounter{theorem}{0}
\setcounter{equation}{0}

\vskip 1.0cm
\begin{center}
\begin{minipage}{0.80\textwidth}
	\begin{center}\textbf{The Discrete Logarithm Problem}\end{center}
	\begin{itemize}
	\item	Let $G$ be a finite cyclic group of order $N \geq 2$, and let $g \in G$ be a generator of $G$.
	\item	Let $h \in G\,\backslash \{\,e\,\}$ be given.
	\item	Find $x \in \N$, $1 \leq x < N$, such that $g^{x} = h$.
	\end{itemize}
\end{minipage}
\end{center}
\vskip 1.0cm

%%%%%%%%%%%%%%%%%%%%%%%%%%%%%%%%%%%%%%%%%%%%%%%%%%%%%%%%%%%%%%%%%%%%%%%%%%%%%%%%%%%%%%%%%%%%%%%%%%%%
\subsection{The Shank Baby-Step-Giant-Step Algorithm}

\noindent
\textbf{Theory:}
\begin{itemize}
\item	Let $n := 1 + \lfloor\,\sqrt{N}\,\rfloor$, where $\lfloor\,\sqrt{N}\,\rfloor$ is the round-down of $\sqrt{N}$.
		Note, trivially, that $n^{2} > N$.
\item	Define:
		\begin{eqnarray*}
		\mathcal{S}_{1} & := & \left\{\,g^{i}\;\vert\;i = 0,1,2,\ldots,n\,\right\}\;\;=\;\;\left\{\;e, g, g^{2}, g^{3}, \ldots, g^{n}\;\right\} \\
		\mathcal{S}_{2} & := & \left\{\,h\cdot(g^{-n})^{i}\;\vert\;i = 0,1,2,\ldots,n\,\right\}\;\;=\;\;\left\{\;h, h\cdot(g^{-n}), h\cdot(g^{-n})^{2}, h\cdot(g^{-n})^{3}, \ldots, h\cdot(g^{-n})^{n}\;\right\}
		\end{eqnarray*}
\item	Then, it turns out that $\mathcal{S}_{1}\cap\mathcal{S}_{2} \neq \varnothing$.
		%Find $i, j \in \{\,0,1,2,\ldots,n\,\}$ such that $g^{i} = h\cdot(g^{-n})^{j} \in \mathcal{S}_{1}\cap\mathcal{S}_{2}$.
\item	For any $i, j \in \{\,0,1,2,\ldots,n\,\}$ such that $g^{i} = h\cdot(g^{-n})^{j} \in \mathcal{S}_{1}\cap\mathcal{S}_{2}$,
		the positive integer $x := i + jn \in \Z$ is a solution to the Discrete Logarithm Problem.
\end{itemize}

\noindent
\textbf{Outline of proof:}
\begin{itemize}
\item	$h \in \langle\,g\,\rangle \backslash \{\,e\,\}$ $\Longrightarrow$ there exists $x \in \N$, $1 \leq x < N$, such that $g^{x} = h$.
\item 	Then, $x = nq + r$, with $0 \leq r < n$ and $q \geq 0$. We claim that the quotient $q$ in fact satisfies: $q < n$.
		Indeed, $x < N$ implies
		\begin{equation*}
		q = \dfrac{x - r}{n} \leq \dfrac{N}{n} < \dfrac{n^{2}}{n} = n.
		\end{equation*}
\item	Hence, $g^{x} = h$ can be rewritten as follows:
		\begin{equation*}
		g^{x} = h
		\;\;\Longleftrightarrow\;\; g^{nq+r} = h
		\;\;\Longleftrightarrow\;\; g^{r} = h\cdot(g^{-n})^{q} \in \mathcal{S}_{1}\cap\mathcal{S}_{2}\,,
		\quad
		\textnormal{for some $0\leq r,q < n$}
		\end{equation*}
\end{itemize}

\noindent
\textbf{Pseudocode:}
\begin{enumerate}
\item	Let $n := 1 + \lfloor\,\sqrt{N}\,\rfloor$, where $\lfloor\,\sqrt{N}\,\rfloor$ is the round-down of $\sqrt{N}$.
\item	Generate $\mathcal{S}_{1} = \left\{\,g^{i}\;\vert\;i = 0,1,2,\ldots,n\,\right\}=\left\{\;e, g, g^{2}, g^{3}, \ldots, g^{n}\;\right\}$.
\item	Form the hash table $\mathcal{T}$:
		\begin{center}
		\begin{tabular}{|c||c|c|c|c|c|c|}
		\hline
		key $g^{i}$ & $e$ & $g$ & $g^{2}$ & $g^{3}$ & $\cdots$ & $g^{n}$ \\
		\hline
		value $i$    & $0$ & $1$ & $2$ & $3$ & $\cdots$ & $n$ \\
		\hline
		\end{tabular}
		\end{center}
\item	\begin{verbatim}
gn := g^{-n};
temp := h;
for j = 0, 1, 2, ...
    if (T{temp} exists) {
        i := T{temp};
        return(i + n*j);
    } else {
        temp := temp * gn;
    }
end
		\end{verbatim}
\end{enumerate}

\noindent
\textbf{Complexity: $\mathcal{O}(\sqrt{N})$}
\begin{itemize}
\item	The generation of $\mathcal{S}_{1}$ takes approximately $n$ multiplications.
\item	The generation of the hash table takes $\mathcal{O}(1)$ computation steps.
\item	The loop takes $\mathcal{O}(n)$ computational steps.
\item	Hence, the whole algorithm takes $\mathcal{O}(n+n+1) = \mathcal{O}(n) = \mathcal{O}(\sqrt{N})$ computational steps.
\end{itemize}

%%%%%%%%%%%%%%%%%%%%%%%%%%%%%%%%%%%%%%%%%%%%%%%%%%%%%%%%%%%%%%%%%%%%%%%%%%%%%%%%%%%%%%%%%%%%%%%%%%%%
\subsection{The Pohlig-Hellman Algorithm}

\noindent
\textbf{Theory:}
\begin{itemize}
\item	The assumption that $h \in G = \langle\,g\,\rangle$ implies the solvability of $g^{x} = h$ for $x \in \{\,0,1,2,\ldots,N-1\,\}$.
\item	Let $N = p_{1}^{\alpha_{1}}\,p_{2}^{\alpha_{2}}\,p_{3}^{\alpha_{3}}\,\cdots\,p_{t}^{\alpha_{t}}$ be the prime factorization of $N$.
		Then, it turns out that the following associated discrete logarithm problems (in finite cyclic groups with prime power order):
		\begin{equation*}
		g_{i}^{y_{i}} \; = \; h_{i},
		\quad\textnormal{where}\quad g_{i} \, := \, g^{N/p_{i}^{\alpha_{i}}},
		\;\;
		h_{i} \, := \, h^{N/p_{i}^{\alpha_{i}}}\,,
		\;\;\textnormal{and}\;\;
		i = 1,2,\ldots,t\,,
		\end{equation*}
		are also solvable for
		\begin{equation*}
		y_{i} \in \{\,0,1,2,\ldots,p_{i}^{\alpha_{i}}-1\,\}\,.
		\end{equation*}
		If we also have their solutions $y_{i}$,  then we can compute $x$ using the Chinese Remainder Theorem,
		namely, by solving the following simultaneous congruences for $x \in \{\,0,1,2,\ldots,N-1\,\}$:
		\begin{equation*}
		x \equiv y_{i}\;\;\textnormal{mod}\;\;p_{i}^{\alpha_{i}}\,,
		\quad\textnormal{for}\;\; i = 1,2,\ldots,t\,.
		\end{equation*}
		\underline{Outline of proof:}\vskip 0.1cm
		\textit{The solvability of $g^{x} = h$ implies the solvability of the auxiliary discrete logarithm problems
		$g_{i}^{y_{i}} = h_{i}$ for $y_{i} \in \{\,0,1,2,\ldots,p_{i}^{\alpha_{i}}-1\,\}$, for each $i = 1,2,\ldots,t$, since
		\begin{equation*}
		g^{x} = h
		\;\;\Longrightarrow\;\; \left(\,g^{x}\,\right)^{N/p_{i}^{\alpha_{i}}} = \left(\,h\,\right)^{N/p_{i}^{\alpha_{i}}}
		\;\;\Longrightarrow\;\; \left(\,g^{N/p_{i}^{\alpha_{i}}}\,\right)^{x} = \left(\,h\,\right)^{N/p_{i}^{\alpha_{i}}}
		\;\;\Longrightarrow\;\; \left(\,g_{i}\,\right)^{x} = h_{i}
		\;\;\Longrightarrow\;\; \left(\,g_{i}\,\right)^{y_{i}} = h_{i}\,,
		\end{equation*}
		where $y_{i} := x \; \textnormal{mod} \; p_{i}^{\alpha_{i}}$.
		Here, note that $\ord(g_{i}) = p_{i}^{\alpha_{i}}$, for each $i = 1,2,\ldots,t$.
		Next, observe that, for each $i = 1,2,\ldots,t$,
		\begin{eqnarray*}
		\left(g^{-x}\cdot h\right)^{N/p_{i}^{\alpha_{i}}}
		&=& \left(g^{-x}\right)^{N/p_{i}^{\alpha_{i}}} \cdot \left(h\right)^{N/p_{i}^{\alpha_{i}}}
		\;\;=\;\; \left(g^{N/p_{i}^{\alpha_{i}}}\right)^{-x} \cdot \left(h\right)^{N/p_{i}^{\alpha_{i}}} \\
		&=& \left(g_{i}\right)^{-x} \cdot h_{i} \\
		&=& \left(g_{i}\right)^{-y_{i}} \cdot h_{i}\,,
			\;\;\textnormal{since}\;\; x \equiv y_{i}\;\;\textnormal{mod}\;\;p_{i}^{\alpha_{i}},
			\;\textnormal{and}\;\;\ord(g_{i}) = p_{i}^{\alpha_{i}} \\
		&=& 1_{G}\,,\;\;\textnormal{since}\;\;g_{i}^{y_{i}} = h_{i}\\
		\end{eqnarray*}
		Hence, $\ord(g^{-x}\cdot h)$ divides $\dfrac{N}{p_{i}^{\alpha_{i}}}$, for each $i = 1,2,\ldots,t$,
		which in turn implies that
		$\ord(g^{-x}\cdot h)$ divides $\gcd\left(\frac{N}{p_{1}^{\alpha_{1}}},\ldots,\frac{N}{p_{t}^{\alpha_{t}}}\right)$.
		However, $\dfrac{N}{p_{1}^{\alpha_{1}}},\ldots,\dfrac{N}{p_{t}^{\alpha_{t}}}$ have no non-trivial common divisors,
		and hence\\ $\gcd\left(\frac{N}{p_{1}^{\alpha_{1}}},\ldots,\frac{N}{p_{t}^{\alpha_{t}}}\right) = 1$.
		Thus, $\ord\left(g^{x}\cdot h\right) = 1$; hence, $g^{x} = h$, as desired.
		}

\end{itemize}

%%%%%%%%%%%%%%%%%%%%%%%%%%%%%%%%%%%%%%%%%%%%%%%%%%%%%%%%%%%%%%%%%%%%%%%%%%%%%%%%%%%%%%%%%%%%%%%%%%%%
\subsection{The Index Calculus Method}



