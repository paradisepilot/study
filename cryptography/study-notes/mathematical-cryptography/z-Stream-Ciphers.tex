
\section{Stream Ciphers}
\setcounter{theorem}{0}
\setcounter{equation}{0}

%%%%%%%%%%%%%%%%%%%%%%%%%%%%%%%%%%%%%%%%
\begin{itemize}
\item
	Pseudorandom number generator:
	\begin{equation*}
	\sigma : \mathcal{K} \longrightarrow \mathcal{S}^{\infty}\,
	\end{equation*}
	In other words, $\sigma$ is a map from the primary key space $\mathcal{K}$
	into the collection $\mathcal{S}^{\infty}$ of infinite sequences of
	elements of the secondary key space $\mathcal{S}$.
	An important required property of $\sigma$ is that each sequence
	$\sigma(k) \in \mathcal{S}^{\infty}$, $k \in \mathcal{K}$,
	is supposed to appear random.
\item
	Character-level encryption:
	\begin{equation*}
	\varepsilon : \Sigma \times \mathcal{S} \longrightarrow \Sigma
	\end{equation*}
	Here, for each fixed secondary key $s \in \mathcal{S}$,
	the map $\varepsilon(\,\cdot\,;s)$ is a permutation of
	the alphabet $\Sigma$.
\item
	Encryption map:
	\begin{equation*}
	E : \Sigma^{n} \times \mathcal{K} \longrightarrow \Sigma^{n}
	\end{equation*}
	\begin{equation*}
	E(m_{1}\,m_{2}\,m_{3} \,\cdots\, m_{n}\,;\,k) \;\; = \;\;
		\varepsilon(m_{1}\,;\sigma_{1}(k))
		\cdot
		\varepsilon(m_{2}\,;\sigma_{2}(k))
		\,\cdots\,
		\varepsilon(m_{n}\,;\sigma_{n}(k)) 
	\end{equation*}
\end{itemize}

%%%%%%%%%%%%%%%%%%%%%%%%%%%%%%%%%%%%%%%%
