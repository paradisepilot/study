
\section{The RSA Cryptosystem}
\setcounter{theorem}{0}
\setcounter{equation}{0}

%%%%%%%%%%%%%%%%%%%%%%%%%%%%%%%%%%%%%%%%%%%%%%%%%%%%%%%%%%%%%%%%%%%%%%%%%%%%%%%%%%%%%%%%%%%%%%%%%%%%
\subsection{Mechanism}

\textbf{Goal:} Alice wants to send Bob an encrypted message through an insecure channel.
\begin{itemize}
\item	Bob chooses his public key $(n,e) \in \N^{2}$ and private key $d \in \N$.
	Bob publishes his public key.
	\begin{itemize}
	\item	$n \in \N$ is called the modulus, with $n = pq$, where $p$ and $q$ are large distinct
		prime numbers.  Note that Bob publishes $n$ but keeps $p$ and $q$ secret.
	\item	$e \in \N$ is the called encryption exponent, and satisfies $\gcd(e,(p-1)(q-1)) = 1$.
	\item	$d \in \N$ is the called decryption exponent, and is determined by $e$ and $n = pq$
		via $d = e^{-1} \in \Z_{(p-1)(q-1)}$.
		Note that $e^{-1} \in \Z_{(p-1)(q-1)}$ exists since $\gcd(e,(p-1)(q-1)) = 1$.
	\end{itemize}
\item	Alice
	\begin{itemize}
	\item	chooses plaintext $m \in \Z_{n} = \Z/n\Z$.
	\item	encrypts her plaintext $m$ using Bob's public key $(n,e)$
		by {\color{red}raising $m \in \Z_{n}$ to the $e^{\textnormal{th}}$ power}.
		In other words, Alice computes her ciphertext $c = m^{e} \in \Z_{n}$.
	\item	sends to Bob through the insecure channel the ciphertext $c \in \Z_{n}$.
	\end{itemize}
\item	Bob decrypts the ciphertext $c \in \Z_{n}$ from Alice by
	{\color{red}taking the $e^{\textnormal{th}}$ root}
	of $c$ in $\Z_{n}$ using his private key $d \in \N$ as follows:
	\begin{equation*}
	c^{d} = \left(m^{e}\right)^{d} = m^{ed} = m^{1+k(p-1)(q-1)} = m \cdot (m^{(p-1)(q-1)})^{k}
	= m \cdot (1)^{k} = m \in \Z_{n}
	\end{equation*}
	\begin{itemize}
	\item	The second last equality follows from $m^{(p-1)(q-1)} \equiv 1 \mod n$, which 
		follows immediately from Euler's Theorem.  It can also be justified with
		Fermat's Little Theorem as follows:
		\begin{equation*}
		\textnormal{Fermat's Little Theorem}
		\;\;\Longrightarrow\;\;
		\left\{\begin{array}{l}
		m^{(p-1)(q-1)} = \left(m^{p-1}\right)^{q-1} \equiv 1 \mod p, \;\;\textnormal{and} \\
		m^{(p-1)(q-1)} = \left(m^{q-1}\right)^{p-1} \equiv 1 \mod q.
		\end{array}\right.
		\end{equation*}
		Hence, $m^{(p-1)(q-1)} - 1$ is divisble by both $p$ and $q$, and hence also by
		$pq = n$ (since $p$ and $q$ are distinct primes).
		Thus, $m^{(p-1)(q-1)} \equiv 1 \mod n = pq$.
	\end{itemize}
\end{itemize}

%%%%%%%%%%%%%%%%%%%%%%%%%%%%%%%%%%%%%%%%%%%%%%%%%%%%%%%%%%%%%%%%%%%%%%%%%%%%%%%%%%%%%%%%%%%%%%%%%%%%
\subsection{Comments}

\begin{itemize}
\item	One-way function (easy): Exponentiation in $\Z_{n}$.
	\begin{itemize}
	\item	Repeating Squaring Algorithm
	\end{itemize}
\item	(Difficult) inverse function: Taking roots in $\Z_{n}$, for $n = pq$, where $p$ and $q$
	are large distinct prime numbers.
\item	Trapdoor: If the factorization of $n = pq$ is known, then we can convert the inverse
	function (taking roots in $\Z_{n}$, which is slow) to an exponentiation in $\Z_{n}$,
	which is fast.
\end{itemize}

%%%%%%%%%%%%%%%%%%%%%%%%%%%%%%%%%%%%%%%%%%%%%%%%%%%%%%%%%%%%%%%%%%%%%%%%%%%%%%%%%%%%%%%%%%%%%%%%%%%%
\subsection{How to find large prime numbers?}

\begin{itemize}
\item	Generate random numbers $x$ with $2^{1023} < x < 2^{1024}$.
\item	The Prime Number Theorem (from Analytic Number Theory) implies that, for large values of $N$,
	the probability that a randomly selected integer $x \in (2^{N-1},2^{N})$ is prime is
	approximately
	\begin{equation*}
	\dfrac{1}{\ln(2^{N})}
	\end{equation*}
\item	Test for the compositeness or ``probable primality" of $x$
	using the Miller-Rabin primality test.
\end{itemize}

