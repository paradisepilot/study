
\section{Perfect Secrecy \& Vernam's One-Time Pad}
\setcounter{theorem}{0}
\setcounter{equation}{0}

%%%%%%%%%%%%%%%%%%%%%%%%%%%%%%%%%%%%%%%%
\subsection{Crytosystems}

\begin{definition}[Cryptosystem]
\mbox{}
\vskip 0.1cm
\noindent
A \,\textbf{cryptosystem}\, is a tuple \,$(\mathcal{T},\mathcal{C},\mathcal{K},E,D)$\,, where
\begin{itemize}
\item
	$\mathcal{T}$, $\mathcal{C}$, and $\mathcal{K}$ are non-empty sets,
\item
	$E : \mathcal{T} \times \mathcal{K} \longrightarrow \mathcal{C}$
	is a surjective map, and
\item
	$D : \mathcal{C} \times \mathcal{K} \longrightarrow \mathcal{T}$\,
	is a map that satisfies:
	For each $k_{1} \in \mathcal{K}$, there exists $k_{2} \in \mathcal{K}$
	such that
	\begin{equation*}
	D\!\left(\,\overset{{\color{white}.}}{E}(t,k_{1})\,,\,k_{2}\,\right) \;\; = \;\; t\,,
	\quad
	\textnormal{for each \,$t \in \mathcal{T}$}
	\end{equation*}
\end{itemize}
We shall also use the notation: For each $k \in \mathcal{K}$,
\begin{equation*}
E_{k} := E(\,\cdot\,,k) : \mathcal{T} \longrightarrow \mathcal{C}
\quad\textnormal{and}\quad
D_{k} := E(\,\cdot\,,k) : \mathcal{C} \longrightarrow \mathcal{T}
\end{equation*}
\end{definition}

%%%%%%%%%%%%%%%%%%%%%%%%%%%%%%%%%%%%%%%%
\subsection{Perfect secrecy of a Shannon cryptosystem}

\begin{definition}[Shannon Cryptosystem]
\mbox{}
\vskip 0.1cm
\noindent
A \,\textbf{Shannon cryptosystem}\, is a tuple \,$(\mathcal{T},\mathcal{C},\mathcal{K},E,D,\mu)$\,,
where
\begin{itemize}
\item
	$(\mathcal{T},\mathcal{C},\mathcal{K},E,D)$ is a cryptosystem,
\item
	$\mathcal{T}$, $\mathcal{K}$ and $\mathcal{C}$ are finite sets,
\item
	$\mu$ is a probability measure defined on the power set $\mathcal{P}(\mathcal{T}\times\mathcal{F})$
	of $\mathcal{T}\times\mathcal{K}$ satisfying
	\begin{equation*}
	P_{\mu}\!\left(\,\overset{{\color{white}.}}{T} = t\,\right)
	\, := \,\mu\!\left(\, \{\,t\,\} \overset{{\color{white}.}}{\times} K \,\right)
	\, > \, 0,
	\quad
	\textnormal{for each $t \in \mathcal{T}$}\,,
	\end{equation*}
\item
	the encryption map $E : \mathcal{T} \times \mathcal{K} \longrightarrow \mathcal{C}$ is
	$\left(
	\overset{{\color{white}.}}{\mathcal{P}}(\mathcal{T}\times\mathcal{K}),\mathcal{P}(\mathcal{C})
	\right)$-measurable, where $\mathcal{P}(\mathcal{C})$ is the power set of $\mathcal{C}$.
\end{itemize}
\end{definition}

\begin{definition}[Perfect secrecy of a Shannon cryptosystem]
\mbox{}
\vskip 0.1cm
\noindent
A Shannon cryptosystem
$(\mathcal{T},\mathcal{C},\mathcal{K},E,D,\mu)$
is said to have \textbf{perfect secrecy} if
\begin{equation*}
P_{\mu}\!\left(\,\left.\overset{{\color{white}.}}{T} = t \;\right\vert C = c\,\right)
\;\; = \;\;
	P_{\mu}\!\left(\,\overset{{\color{white}.}}{T} = t\,\right),
\quad
\textnormal{for each \,$t \in \mathcal{T}$, and for each \,$c \in \mathcal{C}$\, with \,$\mu\!\left(E^{-1}(c)\right) > 0$}\,,
\end{equation*}
where
\begin{eqnarray*}
P_{\mu}\!\left(\,\left.\overset{{\color{white}.}}{T} = {\color{red}t} \;\right\vert C = c\,\right)
& := &
	\dfrac{
		\mu\!\left(\,
			\left(\{\,{\color{red}t}\,\} \overset{{\color{white}.}}{\times} K\right) \,\bigcap\, E^{-1}(c)
			\,\right)
		}{
		\mu\!\left(\, \overset{{\color{white}.}}{E^{-1}}(c) \,\right)
		}
\;\; = \;\;
	\dfrac{
		\mu\!\left(\left\{\,
			\left.
			(\overset{{\color{white}.}}{{\color{red}t}},k) \in \mathcal{T} \times \mathcal{K}
			\;\right\vert\;
			E({\color{red}t},k) = c
			\,\right\}\right)
		}{
		\mu\!\left(\left\{\,
			\left.
			(\overset{{\color{white}.}}{t^{\prime}},k) \in \mathcal{T} \times \mathcal{K}
			\;\right\vert\;
			E(t^{\prime},k) = c
			\,\right\}\right)
		}
\\
P_{\mu}\!\left(\,\overset{{\color{white}.}}{T} = t\,\right)
& := &
	\mu\!\left(\, \{\,t\,\} \overset{{\color{white}.}}{\times} K \,\right)
\end{eqnarray*}
\end{definition}

%%%%%%%%%%%%%%%%%%%%%%%%%%%%%%%%%%%%%%%%
\vskip 0.5cm
\begin{proposition}
\mbox{}
\vskip 0.1cm
\noindent
If a Shannon cryptosystem
$(\mathcal{T},\mathcal{C},\mathcal{K},E,D,\mu)$
has \textbf{perfect secrecy}, then the following statements hold:
\begin{enumerate}
\item
	$\mathcal{K}_{t,c} \; \neq \; \varnothing$\,,
	for each $t \in \mathcal{T}$
	and
	for each $c \in \mathcal{C}$
	with $P_{\mu}\!\left(\overset{{\color{white}.}}{C}=c\,\right) =
	\mu\!\left(\,\overset{{\color{white}.}}{E^{-1}}(c)\,\right) > 0 $,
	where
	\begin{equation*}
	\mathcal{K}_{t,c} \, := \, \left\{\,\left. k \overset{{\color{white}.}}{\in} \mathcal{K} \;\,\right\vert\, E(t,k) = c\,\right\},
	\quad\textnormal{for each $t\in\mathcal{T}$, $c \in \mathcal{C}$}\,,
	\end{equation*}
\item
	$\mathcal{K}_{t_{1},c} \,\cap\, \mathcal{K}_{t_{2},c} \; = \; \varnothing$\,,\;
	for each $c \in \mathcal{C}$
	with $P_{\mu}\!\left(\overset{{\color{white}.}}{C}=c\,\right) \,=\,
	\mu\!\left(\,\overset{{\color{white}.}}{E^{-1}}(c)\,\right) \,>\, 0$,
	and for each $t_{1}, t_{2} \in \mathcal{T}$ with $t_{1} \neq t_{2}$,
	and
\item
	$\#\!\left(\,\mathcal{T}\,\right) \; \leq \; \#\!\left(\,\mathcal{K}\,\right)$.
\end{enumerate}
\end{proposition}

\proof
\begin{enumerate}
\item
	Observe that
	\begin{eqnarray*}
	\dfrac{
		P_{\mu}\!\left(\mathcal{K}_{t,c}\,\right)
		}{
		P_{\mu}\!\left(\,T = t\,\right)
		}
	& = &
		\dfrac{
			P_{\mu}\!\left(\,T = t, C = c\,\right)
			}{
			P_{\mu}\!\left(\,T = t\,\right)
			}
		\;\; = \;\;
		\dfrac{
			P_{\mu}\!\left(\,T = t \;\vert\, C = c\,\right) \cdot P_{\mu}\!\left(C=c\,\right)
			}{
			P_{\mu}\!\left(\,T = t\,\right)
			}
		\;\; = \;\;
		\dfrac{
			P_{\mu}\!\left(\,T = t \,\right) \cdot P_{\mu}\!\left(C=c\,\right)
			}{
			P_{\mu}\!\left(\,T = t\,\right)
			}
	\\
	& \overset{{\color{white}1}}{=} &
		P_{\mu}\!\left(C = c\,\right)
		\;\; > \;\; 0\,,
	\end{eqnarray*}
	where the third equality follows by perfect secrecy hypothesis.
	The preceding inequality implies $P_{\mu}\!\left(\,\mathcal{K}_{t,c}\,\right) \,>\, 0$,
	which in turn implies $\mathcal{K}_{t,c} \,\neq\, \varnothing$.
\item
	$\mathcal{K}_{t_{1},c} \,\cap\, \mathcal{K}_{t_{2},c} \; \neq \; \varnothing$
	\;$\Longrightarrow$\;
	there exists $k \in \mathcal{K}$ such that $E(t_{1},k) = c = E(t_{2},k)$,
	which contradicts injectivity of
	$E(\,\cdot\,,k) : \mathcal{T} \longrightarrow \mathcal{C}$.
	Thus, we must in fact have
	$\mathcal{K}_{t_{1},c} \,\cap\, \mathcal{K}_{t_{2},c} \; = \; \varnothing$.
\item
	First, note that
	$\mu(\,\mathcal{T} \times \mathcal{K}\,) = 1$,
	$\#\!\left(\,\mathcal{C}\,\right) < \infty$, and
	$\mathcal{T} \times \mathcal{K} \, = \underset{c\,\in\,\mathcal{C}}{\bigcup}\,E^{-1}(c)$.
	Hence, we must have $E^{-1}(c) > 0$, for some $c \in \mathcal{C}$.
	Let such a $c \in \mathcal{C}$ be fixed.
	Then, the preceding two statements we just established together imply
	\begin{equation*}
	\mathcal{K}
	\;\; \supset \,
		\underset{t\,\in\,\mathcal{T}}{\bigsqcup}\, \mathcal{K}_{t,c}
	\end{equation*}
	Thus, $\mathcal{K}$ contains the disjoint union of a family, indexed by $\mathcal{T}$,
	of non-empty subsets of $\mathcal{K}$.
	It now follows that $\#\!\left(\,\mathcal{T}\,\right) \,\leq\, \#\!\left(\,\mathcal{K}\,\right)$.
\end{enumerate}
\qed

%%%%%%%%%%%%%%%%%%%%%%%%%%%%%%%%%%%%%%%%
\begin{theorem}[Shannon]
\mbox{}
\vskip 0.1cm
\noindent
Suppose:
\begin{itemize}
\item
	$(\mathcal{T},\mathcal{C},\mathcal{K},E,D,\mu)$
	is a Shannon cryptosystem, and
\item
	$\#\!\left(\,\mathcal{T}\,\right)$
	\,$=$\,
	$\#\!\left(\,\mathcal{C}\,\right)$
	\,$=$\,
	$\#\!\left(\,\mathcal{K}\,\right)$\,.
\end{itemize}
Then, $(\mathcal{T},\mathcal{C},\mathcal{K},E,D,\mu)$
has \textbf{perfect secrecy} if and only if
\begin{itemize}
\item
	the marginal probability distribution induced by $\mu$ on $\mathcal{K}$ is uniform, and
\item
	for each $t \in \mathcal{T}$ and $c \in \mathcal{C}$,
	there exists a unique $k \in \mathcal{K}$
	such that $E(t,k) = c$.
\end{itemize}
\end{theorem}

%%%%%%%%%%%%%%%%%%%%%%%%%%%%%%%%%%%%%%%%
