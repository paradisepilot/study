
\section{Perfect Secrecy \& Vernam's One-Time Pad}
\setcounter{theorem}{0}
\setcounter{equation}{0}

%%%%%%%%%%%%%%%%%%%%%%%%%%%%%%%%%%%%%%%%
\subsection{Crytosystems}

\begin{definition}[Cryptosystem]
\mbox{}
\vskip 0.1cm
\noindent
A \,\textbf{cryptosystem}\, is a tuple \,$(\mathcal{T},\mathcal{C},\mathcal{K},E,D)$\,, where
\begin{itemize}
\item
	$\mathcal{T}$, $\mathcal{C}$, and $\mathcal{K}$ are non-empty sets,
\item
	$E : \mathcal{T} \times \mathcal{K} \longrightarrow \mathcal{C}$
	is a surjective map, and
\item
	$D : \mathcal{C} \times \mathcal{K} \longrightarrow \mathcal{T}$\,
	is a map that satisfies:
	For each $k_{1} \in \mathcal{K}$, there exists $k_{2} \in \mathcal{K}$
	such that
	\begin{equation*}
	D\!\left(\,\overset{{\color{white}.}}{E}(t,k_{1})\,,\,k_{2}\,\right) \;\; = \;\; t\,,
	\quad
	\textnormal{for each \,$t \in \mathcal{T}$}
	\end{equation*}
\end{itemize}
We shall also use the notation: For each $k \in \mathcal{K}$,
\begin{equation*}
E_{k} := E(\,\cdot\,,k) : \mathcal{T} \longrightarrow \mathcal{C}
\quad\textnormal{and}\quad
D_{k} := E(\,\cdot\,,k) : \mathcal{C} \longrightarrow \mathcal{T}
\end{equation*}
\end{definition}

%%%%%%%%%%%%%%%%%%%%%%%%%%%%%%%%%%%%%%%%
\subsection{Perfect secrecy of a Shannon cryptosystem}

\begin{definition}[Shannon Cryptosystem]
\mbox{}
\vskip 0.1cm
\noindent
A \,\textbf{Shannon cryptosystem}\, is a tuple \,$(\mathcal{T},\mathcal{C},\mathcal{K},E,D,\mu)$\,,
where
\begin{itemize}
\item
	$(\mathcal{T},\mathcal{C},\mathcal{K},E,D)$ is a cryptosystem,
\item
	$\mathcal{T}$, $\mathcal{K}$ and $\mathcal{C}$ are finite sets,
\item
	$\mu$ is a probability measure defined on the power set $\mathcal{P}(\mathcal{T}\times\mathcal{F})$
	of $\mathcal{T}\times\mathcal{K}$ satisfying
	\begin{equation*}
	P_{\mu}\!\left(\,\overset{{\color{white}.}}{T} = t\,\right)
	\, := \,\mu\!\left(\, \{\,t\,\} \overset{{\color{white}.}}{\times} K \,\right)
	\, > \, 0,
	\quad
	\textnormal{for each $t \in \mathcal{T}$}\,,
	\end{equation*}
\item
	the encryption map $E : \mathcal{T} \times \mathcal{K} \longrightarrow \mathcal{C}$ is
	$\left(
	\overset{{\color{white}.}}{\mathcal{P}}(\mathcal{T}\times\mathcal{K}),\mathcal{P}(\mathcal{C})
	\right)$-measurable, where $\mathcal{P}(\mathcal{C})$ is the power set of $\mathcal{C}$.
\end{itemize}
\end{definition}

\begin{definition}[Perfect secrecy of a Shannon cryptosystem]
\mbox{}
\vskip 0.1cm
\noindent
A Shannon cryptosystem
$(\mathcal{T},\mathcal{C},\mathcal{K},E,D,\mu)$
is said to have \textbf{perfect secrecy} if
\begin{equation*}
P_{\mu}\!\left(\,\left.\overset{{\color{white}.}}{T} = t \;\right\vert C = c\,\right)
\;\; = \;\;
	P_{\mu}\!\left(\,\overset{{\color{white}.}}{T} = t\,\right),
\quad
\textnormal{for each \,$t \in \mathcal{T}$, and for each \,$c \in \mathcal{C}$\, with \,$\mu\!\left(E^{-1}(c)\right) > 0$}\,,
\end{equation*}
where
\begin{eqnarray*}
P_{\mu}\!\left(\,\left.\overset{{\color{white}.}}{T} = {\color{red}t} \;\right\vert C = c\,\right)
& := &
	\dfrac{
		\mu\!\left(\,
			\left(\{\,{\color{red}t}\,\} \overset{{\color{white}.}}{\times} K\right) \,\bigcap\, E^{-1}(c)
			\,\right)
		}{
		\mu\!\left(\, \overset{{\color{white}.}}{E^{-1}}(c) \,\right)
		}
\;\; = \;\;
	\dfrac{
		\mu\!\left(\left\{\,
			\left.
			(\overset{{\color{white}.}}{{\color{red}t}},k) \in \mathcal{T} \times \mathcal{K}
			\;\right\vert\;
			E({\color{red}t},k) = c
			\,\right\}\right)
		}{
		\mu\!\left(\left\{\,
			\left.
			(\overset{{\color{white}.}}{t^{\prime}},k) \in \mathcal{T} \times \mathcal{K}
			\;\right\vert\;
			E(t^{\prime},k) = c
			\,\right\}\right)
		}
\\
P_{\mu}\!\left(\,\overset{{\color{white}.}}{T} = t\,\right)
& := &
	\mu\!\left(\, \{\,t\,\} \overset{{\color{white}.}}{\times} K \,\right)
\end{eqnarray*}
\end{definition}

%%%%%%%%%%%%%%%%%%%%%%%%%%%%%%%%%%%%%%%%
\vskip 0.5cm
\begin{proposition}
\mbox{}
\vskip 0.1cm
\noindent
If a Shannon cryptosystem
$(\mathcal{T},\mathcal{C},\mathcal{K},E,D,\mu)$
has \textbf{perfect secrecy}, then the following statements hold:
\begin{enumerate}
\item
	$\mathcal{K}_{t,c} \; \neq \; \varnothing$\,,
	for each $t \in \mathcal{T}$
	and
	for each $c \in \mathcal{C}$
	with $P_{\mu}\!\left(\overset{{\color{white}.}}{C}=c\,\right) =
	\mu\!\left(\,\overset{{\color{white}.}}{E^{-1}}(c)\,\right) > 0 $,
	where
	\begin{equation*}
	\mathcal{K}_{t,c} \, := \, \left\{\,\left. k \overset{{\color{white}.}}{\in} \mathcal{K} \;\,\right\vert\, E(t,k) = c\,\right\},
	\quad\textnormal{for each $t\in\mathcal{T}$, $c \in \mathcal{C}$}\,.
	\end{equation*}
\item
	$\mathcal{K}_{t_{1},c} \,\cap\, \mathcal{K}_{t_{2},c} \; = \; \varnothing$\,,\;
	for each $c \in \mathcal{C}$
	with $P_{\mu}\!\left(\overset{{\color{white}.}}{C}=c\,\right) \,=\,
	\mu\!\left(\,\overset{{\color{white}.}}{E^{-1}}(c)\,\right) \,>\, 0$,
	and for each $t_{1}, t_{2} \in \mathcal{T}$ with $t_{1} \neq t_{2}$.
\item
	For each $c \in \mathcal{C}$ with $P_{\mu}\!\left(\,C = c\,\right) > 0$, we have:
	\begin{equation*}
	\underset{t\,\in\,\mathcal{T}}{\bigsqcup}\; \mathcal{K}_{t,c}
	\; \subset \;\;
	\mathcal{K}
	\end{equation*}
\item
	$\#\!\left(\,\mathcal{T}\,\right) \; \leq \; \#\!\left(\,\mathcal{K}\,\right)$.
\end{enumerate}
\end{proposition}

\proof
\begin{enumerate}
\item
	First, observe that the non-emptiness of
	\begin{equation*}
	\left\{\,\overset{{\color{white}.}}{T}=t,C=c\,\right\}
	\; := \;
		\left\{\,\left. (t,k) \overset{{\color{white}.}}{\in} \mathcal{T}\times\mathcal{K} \;\,\right\vert\, E(t,k) = c\,\right\}
	\; = \;
		\left(\,\{\,t\,\} \overset{{\color{white}.}}{\times} K\,\right) \, \bigcap \, E^{-1}(c)
	\end{equation*}
	will immediately implies the non-emptiness of
	\begin{equation*}
	\mathcal{K}_{t,c} \; := \; \left\{\,\left. k \overset{{\color{white}.}}{\in} \mathcal{K} \;\,\right\vert\, E(t,k) = c\,\right\}.
	\end{equation*}
	Next, note that
	\begin{eqnarray*}
	P_{\mu}\!\left(\,T = t, C = c\,\right)
	\;\, = \;\,
		P_{\mu}\!\left(\,T = t \;\vert\, C = c\,\right) \cdot P_{\mu}\!\left(C=c\,\right)
	\;\, = \;\,
		P_{\mu}\!\left(\,T = t \,\right) \cdot P_{\mu}\!\left(C=c\,\right)
	\;\, > \;\,
		0\,,
	\end{eqnarray*}
	where the second equality follows by perfect secrecy hypothesis.\\
	The preceding inequality implies
	$\left\{\,\overset{{\color{white}.}}{T}=t,C=c\,\right\} \,\neq\, \varnothing$,
	which in turn implies $\mathcal{K}_{t,c} \,\neq\, \varnothing$.
\item
	$\mathcal{K}_{t_{1},c} \,\cap\, \mathcal{K}_{t_{2},c} \; \neq \; \varnothing$
	\;$\Longrightarrow$\;
	there exists $k \in \mathcal{K}$ such that $E(t_{1},k) = c = E(t_{2},k)$,
	which contradicts injectivity of
	$E(\,\cdot\,,k) : \mathcal{T} \longrightarrow \mathcal{C}$.
	Thus, we must in fact have
	$\mathcal{K}_{t_{1},c} \,\cap\, \mathcal{K}_{t_{2},c} \; = \; \varnothing$.
\item
	Immediate by the preceding two statements.
\item
	First, note that
	$\mu(\,\mathcal{T} \times \mathcal{K}\,) = 1$,
	$\#\!\left(\,\mathcal{C}\,\right) < \infty$, and
	$\mathcal{T} \times \mathcal{K} \, = \underset{c\,\in\,\mathcal{C}}{\bigcup}\,E^{-1}(c)$.
	Hence, we must have $E^{-1}(c) > 0$, for some $c \in \mathcal{C}$.
	Let such a $c \in \mathcal{C}$ be fixed.
	Then, by the preceding statement, we have
	\begin{equation*}
	\mathcal{K}
	\;\; \supset \,
		\underset{t\,\in\,\mathcal{T}}{\bigsqcup}\, \mathcal{K}_{t,c}
	\end{equation*}
	Thus, $\mathcal{K}$ contains the disjoint union of a family, indexed by $\mathcal{T}$,
	of non-empty subsets of $\mathcal{K}$.
	It now follows that $\#\!\left(\,\mathcal{T}\,\right) \,\leq\, \#\!\left(\,\mathcal{K}\,\right)$.
\end{enumerate}
\qed

%%%%%%%%%%%%%%%%%%%%%%%%%%%%%%%%%%%%%%%%
\clearpage
\begin{theorem}[Shannon]
\mbox{}
\vskip 0.1cm
\noindent
Suppose:
\begin{itemize}
\item
	$(\mathcal{T},\mathcal{C},\mathcal{K},E,D,\mu)$
	is a Shannon cryptosystem, and
\item
	$\#\!\left(\,\mathcal{T}\,\right)$
	\,$=$\,
	$\#\!\left(\,\mathcal{C}\,\right)$
	\,$=$\,
	$\#\!\left(\,\mathcal{K}\,\right)$\,.
\end{itemize}
Then, $(\mathcal{T},\mathcal{C},\mathcal{K},E,D,\mu)$
has \textbf{perfect secrecy} if and only if
\begin{itemize}
\item
	the marginal probability distribution induced by $\mu$ on $\mathcal{K}$ is uniform, and
\item
	for each $t \in \mathcal{T}$ and $c \in \mathcal{C}$
	with $P_{\mu}\!\left(\,C \overset{{\color{white}.}}{=} c\,\right) > 0$,
	there exists a unique $k \in \mathcal{K}$
	such that $E(t,k) = c$.
\end{itemize}
Furthermore, if $(\mathcal{T},\mathcal{C},\mathcal{K},E,D,\mu)$
has perfect secrecy, then the probability distribution induced on $\mathcal{C}$ by
\begin{equation*}
E
\; : \;
	\left(
		\mathcal{T}\overset{{\color{white}.}}{\times}\mathcal{K}
		\,,\,\mathcal{P}(\mathcal{T}\times\mathcal{K})
		\,,\,\mu
	\right)
\;\longrightarrow\;
	\mathcal{C}
\end{equation*}
is also uniform; in particular, in this case, we in fact must have
\begin{equation*}
P_{\mu}\!\left(\,C \overset{{\color{white}.}}{=} c\,\right)
\;\; = \;\;
	\dfrac{1}{\#\!\left(\,C\,\right)}
\;\; > \;\;
	0\,,
\quad
\textnormal{for each $c \in \mathcal{C}$}.
\end{equation*}
\end{theorem}

\proof
\vskip 0.2cm
\noindent
\underline{($\Longrightarrow$)}
\vskip 0.2cm
\noindent
Suppose $(\mathcal{T},\mathcal{C},\mathcal{K},E,D,\mu)$ has perfect secrecy.
For each $t \in \mathcal{T}$ and $c \in \mathcal{C}$, let
$\mathcal{K}_{t,c} \,:=\, \left\{\,\left. k \overset{{\color{white}.}}{\in} \mathcal{K} \;\,\right\vert\, E(t,k) = c\,\right\}$.

\vskip 0.3cm
\noindent
\textbf{Claim 1:}\quad
$\#\!\left(\,\mathcal{K}_{t,c}\,\right) = 1$, for each $t \in \mathcal{T}$ and each $c \in \mathcal{C}$ with
$P_{\mu}\!\left(\,C=c\,\right) > 0$.
\vskip 0.2cm
\noindent
Proof of Claim 1:\quad
Let $c \in \mathcal{C}$ with $P_{\mu}\!\left(\,C=c\,\right) > 0$ be fixed.
By the preceding Proposition, we have that
$\left\{\,\overset{{\color{white}.}}{K}_{t,c}\,\right\}_{t\in\mathcal{T}}$
is a pairwise disjoint collection, indexed by $\mathcal{T}$, of nonempty subsets of $\mathcal{K}$:
\begin{equation*}
\underset{t\,\in\,\mathcal{T}}{\bigsqcup}\; \mathcal{K}_{t,c}
\; \subset \;\;
\mathcal{K}
\end{equation*}
Thus, we have
\begin{equation*}
\#\!\left(\,\mathcal{T}\,\right)
\;\; = \;\;
	\underset{t\,\in\,\mathcal{T}}{\sum}\; 1
\;\; \leq \;\;
	\underset{t\,\in\,\mathcal{T}}{\sum}\, \#\!\left(\,\mathcal{K}_{t,c}\,\right)
\;\; \leq \;\;
	\#\!\left(\,\mathcal{K}\,\right)
\;\; = \;\;
	\#\!\left(\,\mathcal{T}\,\right),
\end{equation*}
which implies
\begin{equation*}
\underset{t\,\in\,\mathcal{T}}{\sum}\, \#\!\left(\,\mathcal{K}_{t,c}\,\right)
\;\; = \;\;
	\#\!\left(\,\mathcal{T}\,\right),
\end{equation*}
which in turn implies
\begin{equation*}
\#\!\left(\,\mathcal{K}_{t,c}\,\right) \; = \; 1\,,
\quad
\textnormal{for each \,$t \in \mathcal{T}$}
\end{equation*}
This proves Claim 1.

\vskip 0.5cm
\noindent
\textbf{Claim 2:}\quad
$P_{\mu}\!\left(\,K \overset{{\color{white}1}}{=} k\,\right)$ $=$
$P_{\mu}\!\left(\, C \overset{{\color{white}1}}{=} c \,\right)$\,,\;
for each $k \in \mathcal{K}$ and each $c \in \mathcal{C}$ with $P_{\mu}\!\left(\, C \overset{{\color{white}1}}{=} c \,\right) > 0$.
%$\mu$ induces the uniform probability distribution on $\mathcal{K}$.
\vskip 0.2cm
\noindent
Proof of Claim 2:\quad
Let $c \in \mathcal{C}$ with $P_{\mu}\!\left(\,C=c\,\right) > 0$ be fixed.
By Claim 1, $\#\!\left(\,\mathcal{K}_{t,c}\,\right) \; = \; 1$, for each $t \in \mathcal{T}$.
In other words, given $t \in \mathcal{T}$ and $c \in \mathcal{C}$ with $P_{\mu}\!\left(\,C = c\,\right) > 0$,
a unique $k_{\,t,c} \in \mathcal{K}$ is determined via $E(t,k) = c$.
Note that $\mathcal{K}_{t,c} \,=\, \left\{\,k_{\,t,c}\,\right\}$, and by the proof of Claim 1,
we see also that
\begin{equation*}
\mathcal{K}
\;\; = \;\;
	\underset{t\,\in\,\mathcal{T}}{\bigsqcup}\; \mathcal{K}_{t,c}
\;\; = \;\;
	\underset{t\,\in\,\mathcal{T}}{\bigsqcup}\, \left\{\, \overset{{\color{white}.}}{k}_{\,t,c}\, \right\}
\end{equation*}
Hence,
\begin{eqnarray*}
\left\{\,\overset{{\color{white}.}}{T}=t \,, C=c\,\right\}
& = &
	\left(\,\{\,t\,\} \overset{{\color{white}.}}{\times} K\,\right) \; \bigcap \; E^{-1}(c)
\\
& = &
	\left\{\,\left. (t,k) \overset{{\color{white}.}}{\in} \mathcal{T}\times\mathcal{K} \;\,\right\vert\, E(t,k) = c\,\right\}
\;\; = \;\;
	\left\{\,(\,\overset{{\color{white}.}}{t},k_{\,t,c}\,)\,\right\}
\;\; = \;\;
	\left\{\,\overset{{\color{white}.}}{T}=t \,, K=k_{\,t,c}\,\right\}
\end{eqnarray*}
Therefore,
\begin{eqnarray*}
P_{\mu}\!\left(\,T = t\,\right)
& = &
	P_{\mu}\!\left(\,\left. T \overset{{\color{white}1}}{=} t \;\right\vert C = c\,\right)
	\;\; = \;\;
	\dfrac{
		P_{\mu}\!\left(\, T \overset{{\color{white}1}}{=} t \,, C = c\,\right)
		}{
		P_{\mu}\!\left(\, C \overset{{\color{white}1}}{=} c \,\right)
		}
	\;\; = \;\;
	\dfrac{
		P_{\mu}\!\left(\, T \overset{{\color{white}1}}{=} t \,, K = k_{\,t,c}\,\right)
		}{
		P_{\mu}\!\left(\, C \overset{{\color{white}1}}{=} c \,\right)
		}
\\
& = &
	\dfrac{
		P_{\mu}\!\left(\, T \overset{{\color{white}1}}{=} t \,\right)
		\cdot
		P_{\mu}\!\left(\, K \overset{{\color{white}1}}{=} k_{\,t,c}\,\right)
		}{
		P_{\mu}\!\left(\, C \overset{{\color{white}1}}{=} c \,\right)
		}
\end{eqnarray*}
Since $P_{\mu}\!\left(\, T \overset{{\color{white}1}}{=} t \,\right) > 0$,
we may cancel it on both sides of the preceding equality, which then implies
$P_{\mu}\!\left(\,K \overset{{\color{white}1}}{=} k_{\,t,c}\,\right)$
$=$ $P_{\mu}\!\left(\,C \overset{{\color{white}1}}{=} c\,\right)$,
for each $t \in \mathcal{T}$.
On the other hand, since
\begin{equation*}
\mathcal{K}
\;\; = \;\;
	\underset{t\,\in\,\mathcal{T}}{\bigsqcup}\; \mathcal{K}_{t,c}
\;\; = \;\;
	\underset{t\,\in\,\mathcal{T}}{\bigsqcup}\, \left\{\, \overset{{\color{white}.}}{k}_{\,t,c}\, \right\},
\end{equation*}
it follows that
\begin{equation*}
P_{\mu}\!\left(\,K \overset{{\color{white}1}}{=} k\,\right)
\;\; = \;\;
	P_{\mu}\!\left(\,C \overset{{\color{white}1}}{=} c\,\right),
\quad
\textnormal{for each $k \in \mathcal{K}$}\,.
\end{equation*}
This proves Claim 2.

\vskip 0.5cm
\noindent
\textbf{Claim 3:}\quad
$P_{\mu}\!\left(\,K \overset{{\color{white}1}}{=} k\,\right)$
$=$ $\dfrac{1}{\#\!\left(\,\mathcal{K}\,\right)}$\,,\,
for each $k \in \mathcal{K}$. 
\vskip 0.2cm
\noindent
Proof of Claim 3:\quad
By Claim 2, we see that $P_{\mu}\!\left(\,K \overset{{\color{white}1}}{=} k\,\right)$
is in fact constant in $k \in \mathcal{K}$, from which Claim 3 follows immediately.


\vskip 0.5cm
\noindent
\textbf{Claim 4:}\quad
$P_{\mu}\!\left(\,C \overset{{\color{white}1}}{=} c\,\right)$
$=$ $\dfrac{1}{\#\!\left(\,\mathcal{C}\,\right)}$\,,
for each $c \in \mathcal{C}$.
In particular, we in fact have
$P_{\mu}\!\left(\, C \overset{{\color{white}1}}{=} c \,\right) > 0$,
for each $c \in \mathcal{C}$.
\vskip 0.2cm
\noindent
Proof of Claim 4:\quad
By Claim 2 and Claim 3, we have that
\begin{equation*}
P_{\mu}\!\left(\, C \overset{{\color{white}1}}{=} c \,\right)
\;\; = \;\;
	\dfrac{1}{\#\!\left(\,\mathcal{K}\,\right)}
\;\; = \;\;
	\dfrac{1}{\#\!\left(\,\mathcal{C}\,\right)}\,,
\quad
\textnormal{for each $c \in \mathcal{C}$ with $P_{\mu}\!\left(\, C \overset{{\color{white}1}}{=} c \,\right) > 0$}\,.
\end{equation*}
Lastly, note that
\begin{eqnarray*}
1
& = &
	\underset{c\,\in\,\mathcal{C}}{\sum}\; P_{\mu}\!\left(\, C \overset{{\color{white}1}}{=} c \,\right)
\;\; = \;\;
	\underset{P_{\mu}(C=c)\,>\,0}{\underset{c\;\in\;\mathcal{C}}{\sum}}
	P_{\mu}\!\left(\, C \overset{{\color{white}1}}{=} c \,\right)
\;\; = \;\;
	\underset{P_{\mu}(C=c)\,>\,0}{\underset{c\;\in\;\mathcal{C}}{\sum}}
	\dfrac{1}{\#\!\left(\,\mathcal{C}\,\right)}
\\
& = &
	\dfrac{1}{\#\!\left(\,\mathcal{C}\,\right)}
	\cdot
	\#\!\left(\left\{\,\left. c \overset{{\color{white}.}}{\in} \mathcal{C} \;\,\right\vert\, P_{\mu}(C=c)\,>\,0 \,\right\}\right),
\end{eqnarray*}
which implies that $P_{\mu}\!\left(\,C \overset{{\color{white}.}}{=} c\,\right) > 0$,
for each $c \in \mathcal{C}$.
This proves Claim 4.

\vskip 0.5cm
\noindent
We now complete this direction of the proof by noting that Claims 1, 2, and 3
together yield the two bullets in the Theorem's conclusions, while Claim 4 yields
the ``furthermore'' part of the Theorem's conclusions.

\vskip 1.0cm
\noindent
\underline{($\Longleftarrow$)}\quad
Suppose $(\mathcal{T},\mathcal{C},\mathcal{K},E,D,\mu)$
has \textbf{perfect secrecy}.

\qed

%%%%%%%%%%%%%%%%%%%%%%%%%%%%%%%%%%%%%%%%
