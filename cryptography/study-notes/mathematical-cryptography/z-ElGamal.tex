
\section{The ElGamal Public Key Cryptosystem}
\setcounter{theorem}{0}
\setcounter{equation}{0}

%%%%%%%%%%%%%%%%%%%%%%%%%%%%%%%%%%%%%%%%%%%%%%%%%%%%%%%%%%%%%%%%%%%%%%%%%%%%%%%%%%%%%%%%%%%%%%%%%%%%
\subsection{Mechanism}

\textbf{Goal:} Bob wants to send Alice an encrypted message via an insecure channel.
\begin{itemize}
\item	Alice chooses and publishes a large prime $p$ and an integer $g$ with large prime order in $\F_{p}^{*}$.
\item	Alice chooses a secret $a \in \Z$, with $1 \leq a \leq p - 1$, computes $A := g^{a} \mod p$, and
		publishes $A \in\F_{p}^{*}$ as her public key.
\item	Bob chooses a plaintext $m \in \F_{p}^{*}$, and a secret random ephemeral key $k \in \Z$, with $1 \leq k \leq p - 1$.
		Bob computes $B_{1} := g^{k} \mod p$, and $B_{2} := m\cdot A^{k} \in\F_{p}^{*}$.
		Bob's ciphertext is $(B_{1},B_{2})$ and Bob sends it to Alice via the insecure channel.
\item	Alice decrypts Bob's ciphertext $(B_{1},B_{2})$ by computing $B_{2}\cdot(B_{1}^{a})^{-1} \in \F_{p}^{*}$.
		Note that
		\begin{equation*}
		B_{2}\cdot(B_{1}^{a})^{-1}
		\;\; \equiv \;\; (m\cdot A^{k}) \cdot (g^{ak})^{-1}
		\;\; \equiv \;\; m\cdot (g^{a})^{k} \cdot(g^{ak})^{-1}
		\;\; \equiv \;\; m\cdot (g^{ak}) \cdot(g^{ak})^{-1}
		\;\; \equiv \;\; m \mod p
		\end{equation*}
\end{itemize}

%%%%%%%%%%%%%%%%%%%%%%%%%%%%%%%%%%%%%%%%%%%%%%%%%%%%%%%%%%%%%%%%%%%%%%%%%%%%%%%%%%%%%%%%%%%%%%%%%%%%
\subsection{Security of the ElGamal public key cryptosystem}

\begin{itemize}
\item	The ElGamal public key encryption system is at least as secure as the difficulty of Diffie-Hellman problem,
		in the sense that en ElGamal oracle (efficient solver of the ElGamal cryptosystem problem) can be used
		to efficiently solve the Diffie-Hellman problem.
\end{itemize}

%%%%%%%%%%%%%%%%%%%%%%%%%%%%%%%%%%%%%%%%%%%%%%%%%%%%%%%%%%%%%%%%%%%%%%%%%%%%%%%%%%%%%%%%%%%%%%%%%%%%



