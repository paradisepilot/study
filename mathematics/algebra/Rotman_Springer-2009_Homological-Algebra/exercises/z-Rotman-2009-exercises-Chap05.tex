
          %%%%% ~~~~~~~~~~~~~~~~~~~~ %%%%%

\section{Setting the Stage}
\setcounter{theorem}{0}
\setcounter{equation}{0}

%\cite{vanDerVaart1996}
%\cite{Kosorok2008}

%\renewcommand{\theenumi}{\alph{enumi}}
%\renewcommand{\labelenumi}{\textnormal{(\theenumi)}$\;\;$}
\renewcommand{\theenumi}{\roman{enumi}}
\renewcommand{\labelenumi}{\textnormal{(\theenumi)}$\;\;$}

          %%%%% ~~~~~~~~~~~~~~~~~~~~ %%%%%

\noindent
\textbf{Exercise 5.1}
\begin{enumerate}
\item
	Prove that $\varemptyset$ is an initial object in \Sets.
	
	\proof
	First, recall that given any two sets $X$ and $Y$, a function $f$ from $X$ into $Y$ is,
	by definition, a subset
	\begin{equation*}
	\textnormal{Graph}(f) \;\; \subset \;\; X \times Y
	\end{equation*}
	with the following property:
	for each $x_{0} \in X$, there {\color{red}exists} a {\color{red}unique} $y_{0} \in Y$ such that
	$(x_{0},y_{0}) \in \textnormal{Graph}(f) \subset X \times Y$.
	
	Thus, given $X, Y \in \Obj(\Sets)$, the set $\Hom_{\,\Sets}(X,Y)$ of all
	functions (morphisms in \Sets) from $X$ to $Y$ is
	\begin{equation*}
	\Hom_{\,\Sets}(X,Y)
	\;\; := \;\;
		\left\{\;
			\mathcal{G} \in \mathcal{P}(X \times Y)
			\;\left\vert\,
			\begin{array}{c}
				\textnormal{for each $x_{0} \in X$,}
				\\
				\textnormal{$\exists$\, unique $y_{0} \in Y$ such that $(x_{0},y_{0}) \in \mathcal{G}$}
			\end{array}
			\right.
		\right\}
	\end{equation*}
	In particular, for any set $Y$, we have $\Hom_{\,\Sets}(\varemptyset,Y) \,=\, \left\{\,\varemptyset\,\right\}$,
	since
	\begin{equation*}
	\mathcal{G} \in \Hom_{\,\Sets}(\varemptyset,Y)
	\quad\Longrightarrow\quad
		\varemptyset \,\subset\, \mathcal{G} \,\subset\, \mathcal{P}(\varemptyset \times Y) \,=\, \varemptyset
	\quad\Longrightarrow\quad
		\mathcal{G} = \varemptyset
	\end{equation*}
	This shows that $\Hom_{\,\Sets}(\varemptyset,Y) \,=\, \left\{\,\varemptyset\,\right\}$ indeed
	has a unique element, for each $Y \in \Obj(\Sets)$, and hence $\varemptyset \in \Obj(\Sets)$ is indeed
	an initial object in \Sets.

\item
	Prove that any one-point set \,$\Omega = \{\,\omega_{0}\,\}$\, is a terminal object in \Sets.\\
	In particular, what is the function $\varemptyset \longrightarrow \Omega$\,?
	
	\proof
	It is clear that, for each set $X \in \Obj(\Sets)$, each function $f : X \longrightarrow \{\,\omega_{0}\,\}$
	with domain $X$ must map each element of $X$ to $\omega_{0}$, which immediately implies
	\begin{equation*}
	\Hom_{\,\Sets}\!\left(\overset{{\color{white}.}}{X},\{\,\omega_{0}\,\}\right)
	\;\; = \;\;
		\left\{\; \overset{{\color{white}.}}{f}_{X,\,\omega_{0}} \;\right\},
	\end{equation*}
	where \,$f_{X,\,\omega_{0}} : X \longrightarrow \{\,\omega_{0}\,\}$\,
	is the constant function with value $\omega_{0}$ defined on $X$.
	This proves that the (arbitrary) one-point set \,$\{\,\omega_{0}\,\}$\, is indeed a terminal object in \Sets.
	
	In particular,
	\begin{eqnarray*}
	\Hom_{\,\Sets}\!\left(\,\overset{{\color{white}.}}{\varemptyset},\{\,\omega_{0}\,\}\,\right)
	& := &
		\left\{\;
			\mathcal{G} \in \mathcal{P}(\varemptyset \times \{\,\omega_{0}\,\})
			\;\left\vert\,
			\begin{array}{c}
				\textnormal{for each $x_{0} \in \varemptyset$,}
				\\
				\textnormal{$\exists$\, unique $y_{0} \in \{\,\omega_{0}\,\}$ such that $(x_{0},y_{0}) \in \mathcal{G}$}
			\end{array}
			\right.
		\right\}
	\\
	& \overset{{\color{white}\vert}}{=} &
		\mathcal{P}(\varemptyset \times \{\,\omega_{0}\,\})
	\;\; = \;\;
		\left\{\,\varemptyset\,\right\}
	\end{eqnarray*}

\end{enumerate}

          %%%%% ~~~~~~~~~~~~~~~~~~~~ %%%%%

\vskip 0.5cm
\noindent
\textbf{Exercise 5.2}
\vskip 0.2cm
\noindent
A \,\textit{zero object}\, in a category $\mathcal{C}$ is an object that is both an initial object
and a terminal object.
\begin{enumerate}
\item
	Prove the uniqueness up to isomorphism of initial, terminal, and zero objects, if they exist.
	
	\proof
	\vskip 0.2cm
	\underline{Initial objects, if they exist, are unique up to isomorphism.}
	\vskip 0.0cm
	Suppose $I_{1}, I_{2} \in \Obj(\mathcal{C})$ be two initial objects in the category $\mathcal{C}$.
	Thus, each of $\Hom_{\,\mathcal{C}}(I_{1},I_{1})$, $\Hom_{\,\mathcal{C}}(I_{2},I_{2})$,
	$\Hom_{\,\mathcal{C}}(I_{1},I_{2})$ and $\Hom_{\,\mathcal{C}}(I_{2},I_{1})$ is a singleton set.
	In particular, we have
	\begin{equation*}
	\Hom_{\,\mathcal{C}}(I_{1},I_{1}) \,=\, \left\{\;\overset{{\color{white}.}}{\id}_{I_{1}}\right\}
	\quad\textnormal{and}\quad
	\Hom_{\,\mathcal{C}}(I_{2},I_{2}) \,=\, \left\{\;\overset{{\color{white}.}}{\id}_{I_{2}}\,\right\}.
	\end{equation*}
	Next, write
	\begin{equation*}
	\Hom_{\,\mathcal{C}}(I_{1},I_{2}) \,=\, \left\{\;\overset{{\color{white}.}}{f}\,\right\}
	\quad\textnormal{and}\quad
	\Hom_{\,\mathcal{C}}(I_{2},I_{1}) \,=\, \left\{\;\overset{{\color{white}-}}{g}\,\right\}.
	\end{equation*}
	Then, note that
	\begin{equation*}
	f \circ g \;\in\; \Hom_{\,\mathcal{C}}(I_{2},I_{2}) \;=\; \left\{\;\overset{{\color{white}.}}{\id}_{I_{2}}\,\right\}
	\quad\textnormal{and}\quad
	g \circ f \;\in\; \Hom_{\,\mathcal{C}}(I_{1},I_{1}) \;=\; \left\{\;\overset{{\color{white}.}}{\id}_{I_{1}}\,\right\}
	\end{equation*}
	Thus, \,$f$\, and \,$g$\, are isomorphisms, and are inverses of each other.
	This proves that initial objects are indeed unique up to isomorphism.

	\vskip 0.2cm
	\underline{Terminal objects, if they exist, are unique up to isomorphism.}
	\vskip 0.0cm

\item
	Prove that \,$\left\{\,0\,\right\}$\, is a zero object in $_{R}\Mod$ and that
	\,$\left\{\,1\,\right\}$\, is a zero object in \Groups.
	
\item
	Prove that neither \Sets\, nor \Top\, has a zero object.
	
\end{enumerate}

          %%%%% ~~~~~~~~~~~~~~~~~~~~ %%%%%

%\renewcommand{\theenumi}{\alph{enumi}}
%\renewcommand{\labelenumi}{\textnormal{(\theenumi)}$\;\;$}
\renewcommand{\theenumi}{\roman{enumi}}
\renewcommand{\labelenumi}{\textnormal{(\theenumi)}$\;\;$}

          %%%%% ~~~~~~~~~~~~~~~~~~~~ %%%%%
