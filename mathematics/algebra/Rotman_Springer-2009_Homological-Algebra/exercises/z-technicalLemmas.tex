
          %%%%% ~~~~~~~~~~~~~~~~~~~~ %%%%%

\section{Technical lemmas}
\setcounter{theorem}{0}
\setcounter{equation}{0}

%\cite{vanDerVaart1996}
%\cite{Kosorok2008}

%\renewcommand{\theenumi}{\alph{enumi}}
%\renewcommand{\labelenumi}{\textnormal{(\theenumi)}$\;\;$}
\renewcommand{\theenumi}{\roman{enumi}}
\renewcommand{\labelenumi}{\textnormal{(\theenumi)}$\;\;$}

          %%%%% ~~~~~~~~~~~~~~~~~~~~ %%%%%

\begin{lemma}[Markov's Inequality]
\label{lemma:MarkovInequality}
\mbox{}\vskip 0.2cm
\noindent
Suppose \,$Z : (\Omega,\mathcal{A},\mu) \longrightarrow [0,\infty)$\,
is a non-negative $\Re$-valued random variable.
Then,
\begin{equation*}
P\!\left(\; Z \,\overset{{\color{white}.}}{\geq}\, \alpha \;\right)
\;\; \leq \;\;
	\dfrac{ E\!\left[\; Z \;\right] }{ \alpha }\,,
\quad
\textnormal{for each \,$\alpha > 0$}
\end{equation*}
\end{lemma}
\proof
First, observe that
\begin{equation*}
\alpha \cdot I_{\{Z \,\geq\, \alpha\}} \;\; \leq \;\; Z\,,
\quad
\textnormal{for each \,$\alpha > 0$}
\end{equation*}
Taking expectation on both sides yields:
\begin{equation*}
E\!\left[\; \alpha \overset{{\color{white}-}}{\cdot} I_{\{Z \,\geq\, \alpha\}} \;\right]
\;\; \leq \;\;
	E\!\left[\; Z \;\right]\,,
\quad
\textnormal{for each \,$\alpha > 0$}\,,
\end{equation*}
which immediately implies:
\begin{equation*}
E\!\left[\; I_{\{Z \,\overset{{\color{white}.}}{\geq}\, \alpha\}} \;\right]
\;\; \leq \;\;
	\dfrac{ E\!\left[\; Z \;\right] }{ \alpha }\,,
\quad
\textnormal{for each \,$\alpha > 0$}\,.
\end{equation*}
Hence,
\begin{equation*}
P\!\left(\; Z \,\overset{{\color{white}.}}{\geq}\, \alpha \;\right)
\;\; = \;\;
	E\!\left[\; I_{\{Z \,\overset{{\color{white}.}}{\geq}\, \alpha\}} \;\right]
\;\; \leq \;\;
	\dfrac{ E\!\left[\; Z \;\right] }{ \alpha }\,,
\quad
\textnormal{for each \,$\alpha > 0$}\,,
\end{equation*}
as required.
\qed

          %%%%% ~~~~~~~~~~~~~~~~~~~~ %%%%%

\vskip 1.0cm
\begin{corollary}
\label{corollary:MarkovInequality}
\mbox{}\vskip 0.2cm
\noindent
Suppose \,$Z : (\Omega,\mathcal{A},\mu) \longrightarrow [0,1]$\,
is a random variable. Let \,$\mu_{Z} \,:=\, E\!\left[\,Z\,\right]$.
Then,
\begin{equation*}
P\!\left(\,Z \,\overset{{\color{white}.}}{>}\, \varepsilon\,\right)
\;\; \geq \;\;
	\dfrac{\mu_{Z} \,-\, \varepsilon}{1\,-\,\varepsilon}\,,
\quad
\textnormal{for each \,$\varepsilon \in (0,1)$}
\end{equation*}
In particular
\end{corollary}
\proof
Let \,$Y \,:=\, 1 - Z$.\, Then, \,$Y$\, is a non-negative $\Re$-valued random variable.
By Markov's Inequality (Lemma \ref{lemma:MarkovInequality}), we have
\begin{equation*}
P\!\left(\; Y \,\overset{{\color{white}.}}{\geq}\, \alpha \;\right)
\;\; \leq \;\;
	\dfrac{ E\!\left[\; Y \;\right] }{ \alpha }\,,
\quad
\textnormal{for each \,$\alpha > 0$}
\end{equation*}
In terms of \,$Z$,\, we thus have
\begin{equation*}
P\!\left(\; Z \,\overset{{\color{white}.}}{\leq}\, 1 - \alpha \;\right)
\;\; = \;\;
	P\!\left(\; 1 - Z \,\overset{{\color{white}.}}{\geq}\, \alpha \;\right)
\;\; = \;\;
	P\!\left(\; Y \,\overset{{\color{white}.}}{\geq}\, \alpha \;\right)
\;\; \leq \;\;
	\dfrac{ E\!\left[\; Y \;\right] }{ \alpha }
\;\; = \;\;
	\dfrac{ 1 - E\!\left[\; Z \;\right] }{ \alpha }\,,
\quad
\textnormal{for each \,$\alpha > 0$}
\end{equation*}
Restricting to \,$\alpha \in (0,1)$,\, and letting \,$\varepsilon \,:=\, 1 - \alpha \in (0,1)$,\, we have
\begin{equation*}
P\!\left(\; Z \,\overset{{\color{white}.}}{\leq}\, \varepsilon \;\right)
\;\; \leq \;\;
	\dfrac{ 1 \,-\, \mu_{Z} }{ 1 \,-\, \varepsilon }\,,
\quad
\textnormal{for each \,$\varepsilon \in (0,1)$}
\end{equation*}
Equivalently,
\begin{equation*}
P\!\left(\; Z \,\overset{{\color{white}.}}{>}\, \varepsilon \;\right)
\;\; = \;\;
	1 \,-\, P\!\left(\; Z \,\overset{{\color{white}.}}{\leq}\, \varepsilon \;\right)
\;\; \geq \;\;
	1 \,-\, \dfrac{ 1 \,-\, \mu_{Z} }{ 1 \,-\, \varepsilon }
\;\; = \;\;
	\dfrac{ \mu_{Z} \,-\, \varepsilon }{ 1 \,-\, \varepsilon }\,,
\quad
\textnormal{for each \,$\varepsilon \in (0,1)$}\,,
\end{equation*}
as required.
\qed

          %%%%% ~~~~~~~~~~~~~~~~~~~~ %%%%%

\vskip 1.0cm
\begin{lemma}[Hoeffding's Lemma]
\label{lemma:HoeffdingLemma}
\mbox{}\vskip 0.2cm
\noindent
Let \,$X : (\Omega,\mathcal{A}) \longrightarrow \Re$\, be an
$\Re$-valued random variable and $a,b \in \Re$, with $a < b$, such that
\begin{equation*}
P\!\left(\;a \leq X \leq b\;\right) \,=\, 1\,,
\quad\textnormal{and}\quad
E\!\left[\,X\,\right] \,=\, 0\,.
\end{equation*}
Then,
\begin{equation*}
E\!\left[\;\overset{{\color{white}-}}{\exp}\!\left(\,\lambda\,X\,\right)\;\right]
	\;\; \leq \;\;
	\exp\!\left(\,\dfrac{\lambda^{2}\,(b-a)^{2}}{8}\,\right)\,,
\quad
\textnormal{for each \,$\lambda > 0$}\,.
\end{equation*}
\end{lemma}
\proof
Since \,$x \longmapsto \exp\!\left(\,\lambda\,x\,\right)$\, is a convex function,
we have
\begin{equation*}
\exp\!\left(\,\lambda\,x\,\right)
\;\; \leq \;\;
	\alpha \cdot \exp\!\left(\,\lambda\,a\,\right)
	\, + \,
	(1-\alpha) \cdot \exp\!\left(\,\lambda\,b\,\right)\,,
\quad
\textnormal{for each \,$x \in [\,a,b\,]$\, and \,$\alpha \in (0,1)$}\,.
\end{equation*}
Setting \,$\alpha = \dfrac{b - x}{b - a}$\, yields
\begin{equation*}
\exp\!\left(\,\lambda\,x\,\right)
\;\; \leq \;\;
	\dfrac{b-x}{b-a} \cdot \exp\!\left(\,\lambda\,a\,\right)
	\, + \,
	\dfrac{x-a}{b-a} \cdot \exp\!\left(\,\lambda\,b\,\right)
\end{equation*}
Taking expectation now gives
\begin{equation*}
E\!\left[\;\overset{{\color{white}-}}{\exp}\!\left(\,\lambda\,X\,\right)\;\right]
\;\; \leq \;\;
	\dfrac{b-E\!\left[\,X\,\right]}{b-a} \cdot \exp\!\left(\,\lambda\,a\,\right)
	\, + \,
	\dfrac{E\!\left[\,X\,\right]-a}{b-a} \cdot \exp\!\left(\,\lambda\,b\,\right)
\;\; = \;\;
	\dfrac{b}{b-a} \cdot \exp\!\left(\,\lambda\,a\,\right)
	\, - \,
	\dfrac{a}{b-a} \cdot \exp\!\left(\,\lambda\,b\,\right)\,,
\end{equation*}
since \,$E\!\left[\,X\,\right] \,=\, 0$.\,
Now, let \,$t \,:=\, \lambda\cdot(b-a)$, \;$p \,:=\, \dfrac{-a}{b-a}$\;
and \,$f(t) \,:=\, -p\,t + \log\!\left(\,1-p+p\,e^{t}\,\right)$.
Then,
\begin{equation*}
-\,p\,t \,=\, -\left(\,\dfrac{-a}{b-a}\,\right)\cdot\lambda\cdot(b-a) \,=\, \lambda\,a\,
\quad\quad\textnormal{and}\quad\quad
1 \, - \, p \,=\, \cdots \,=\, \dfrac{b}{b-a} 
\end{equation*}
Hence,
\begin{eqnarray*}
\exp\!\left(\,\overset{{\color{white}.}}{f}(t)\,\right)
&=&
	\exp\!\left(\;
		-\,p\,t \,\overset{{\color{white}.}}{+}\, \log(\,1-p+p\,e^{t}\,)
		\;\right)
\;\; = \;\;
	\exp\!\left(\;-\,p\,t \;\right) \cdot \left[\; 1 \,-\, p \,\overset{{\color{white}.}}{+}\, p\,e^{t} \;\right]
\\
&=&
	\exp\!\left(\, \lambda\,a \,\right)
	\cdot
	\left[\;\,
		\dfrac{b}{b-a}
		\,\overset{{\color{white}.}}{+}\,
		\left(\dfrac{-a}{b-a}\right)\,\exp(\,\lambda\,(b-a)\,)
		\;\right]
\\
&=&
	\dfrac{b}{b-a} \cdot \exp\!\left(\,\lambda\,a\,\right)
	\, - \,
	\dfrac{a}{b-a} \cdot \exp\!\left(\,\lambda\,b\,\right)
\end{eqnarray*}
We therefore see that
\begin{equation*}
E\!\left[\;\overset{{\color{white}-}}{\exp}\!\left(\,\lambda\,X\,\right)\;\right]
\;\; \leq \;\;
	\dfrac{b}{b-a} \cdot \exp\!\left(\,\lambda\,a\,\right)
	\, - \,
	\dfrac{a}{b-a} \cdot \exp\!\left(\,\lambda\,b\,\right)
\;\; = \;\;
	\exp\!\left(\,\overset{{\color{white}.}}{f}(t)\,\right)
\end{equation*}
Thus, to complete the proof of Hoeffding's Lemma, it suffices to show that
\,$f(t) \,\leq\, \dfrac{t^{2}}{8}$,\, for each \,$t \geq 0$.\,
But this last inequality follows easily from elementary Calculus:
First, note that
$f^{\prime}(t)\,=\,-\,p+\dfrac{p\,e^{t}}{1-p+p\,e^{t}}$\,,\, and
\begin{eqnarray*}
f^{\prime\prime}(t)
& = &
	\dfrac{p\,e^{t}}{1 - p + p\,e^{t}}
	\,-\,
	\dfrac{p\,e^{t} \cdot p\,e^{t}}{(\,1 - p + p\,e^{t}\,)^{2}}
\;\; = \;\;
	\dfrac{p\,e^{t}}{1 - p + p\,e^{t}}
	\cdot
	\left(\,1 \,-\, \dfrac{p\,e^{t}}{1 - p + p\,e^{t}}\,\right)
\\
& \leq &
	\underset{\zeta\,\in\,\Re}{\sup}\;\left\{\,\zeta\,(1\overset{{\color{white}.}}{-}\zeta)\,\right\}
\;\; = \;\;
	\dfrac{1}{4}\,,
	\quad\textnormal{for each \,$t \in \Re$}
\end{eqnarray*}
Since \,$f^{\prime}(0) = 0$,\, we have
\begin{equation*}
f^{\prime}(t)
\;\; = \;\;
	f^{\prime}(t) \,-\, 0
\;\; = \;\;
	f^{\prime}(t) \,-\, f^{\prime}(0)
\;\; = \;\;
	\int^{t}_{0}\; f^{\prime\prime}(\zeta) \;\d\,\zeta
\;\; \leq \;\;
	\int^{t}_{0}\;\dfrac{1}{4}\;\d\,\zeta
\;\; = \;\;
	\dfrac{t}{4}\,,
\quad\textnormal{for each \,$t \geq 0$}.
\end{equation*}
Since \,$f(0) = 0$,\, we have
\begin{equation*}
f(t)
\;\; = \;\;
	f(t) \,-\, 0
\;\; = \;\;
	f(t) \,-\, f(0)
\;\; = \;\;
	\int^{t}_{0}\; f^{\prime}(\zeta) \;\d\,\zeta
\;\; \leq\;\;
	\int^{t}_{0}\;\dfrac{\zeta}{4}\;\d\,\zeta
\; = \;
	\dfrac{t^{2}}{8}\,,
\quad\textnormal{for each \,$t \geq 0$}.
\end{equation*}
This completes the proof of Hoeffding's Lemma.
\qed

          %%%%% ~~~~~~~~~~~~~~~~~~~~ %%%%%

\vskip 1.0cm
\begin{lemma}[Hoeffding's Inequality]
\label{lemma:HoeffdingInequality}
\mbox{}\vskip 0.2cm
\noindent
Suppose:
\begin{itemize}
\item
	$Z : (\Omega,\mathcal{A},\mu) \longrightarrow \Re$\,
	is an $\Re$-valued random variable.
\item
	There exist \,$a < b \in \Re$\, such that \,$P\!\left(\,a \leq Z \leq b\,\right) \,=\, 1$.
\end{itemize}
Then, we have
\begin{equation*}
P\!\left(\;\,\left\vert\,
	\dfrac{1}{n}\;\overset{n}{\underset{i=1}{\sum}}\,Z_{i} \, - \, \mu
	\,\right\vert
	\,>\,\varepsilon
	\;\right)
\;\; \leq \;\;
	2\,\exp\!\left(\;
		-\,\dfrac{
			2\,n\,\varepsilon^{2}
			}{
			(b-a)^{2}
			}
		\;\right)\,,
\quad
\textnormal{for each \,$\varepsilon > 0$}\,,
\end{equation*}
where
\,$\mu \, := \, E\!\left[\,Z\,\right]$\, and
\,$Z_{1}, Z_{2}, \ldots, Z_{n} : (\Omega,\mathcal{A},\mu) \longrightarrow \Re$\,
are I.I.D. copies of $Z$.
\end{lemma}
\proof
First, note that for any \,$\lambda > 0$,\, we have
\begin{eqnarray*}
P\!\left(\;\,
	\dfrac{1}{n}\;\overset{n}{\underset{i=1}{\sum}}\,Z_{i} \, - \, \mu
	\,>\,\varepsilon
	\;\right)
& \leq &
	P\!\left(\;\,
		\dfrac{1}{n}\;\overset{n}{\underset{i=1}{\sum}}\,Z_{i} \, - \, \mu
		\,\geq\,\varepsilon
		\;\right)
\;\; = \;\;
	P\!\left(\;\,
		\dfrac{1}{n}\;\overset{n}{\underset{i=1}{\sum}}\,(Z_{i} \, - \, \mu)
		\,\geq\,\varepsilon
		\;\right)
\\
& = &
	P\!\left(\;
		\exp\!\left(\,\dfrac{{\color{red}\lambda}}{n}\;\overset{n}{\underset{i=1}{\sum}}\,(Z_{i} \, - \, \mu)\,\right)
		\,\geq\, \exp(\,{\color{red}\lambda}\,\varepsilon\,)
		\;\right),
	\quad
	\textnormal{by $\lambda > 0$ and monotonicity of \,$\exp(\,\cdot\,)$}
\\
& \leq &
	\left.E\!\left[\;
		\exp\!\left(\,\dfrac{\lambda}{n}\;\overset{n}{\underset{i=1}{\sum}}\,(Z_{i} \, - \, \mu)\,\right)
		\;\right]
		\right\slash \exp(\,\lambda\,\varepsilon\,)\,,
	\quad
	\textnormal{by Markov's Inequality, Lemma \ref{lemma:MarkovInequality}}
%\\
%& = &
%	\left.E\!\left[\;\,
%		\overset{n}{\underset{i=1}{\prod}}\;
%		\exp\!\left(\,\dfrac{\lambda}{n}\;(Z_{i} \, - \, \mu)\,\right)
%		\;\right]
%		\right\slash \exp(\,\lambda\,\varepsilon\,)
\\
& = &
	\overset{n}{\underset{i=1}{\prod}}\;
	\left.E\!\left[\;\,
		\exp\!\left(\,\dfrac{\lambda}{n}\;(Z_{i} \, - \, \mu)\,\right)
		\;\right]
		\right\slash \exp(\,\lambda\,\varepsilon\,)\,,
	\quad
	\textnormal{by independence of the $Z_{i}$'s}
\\
& \leq &
	\overset{n}{\underset{i=1}{\prod}}\;
	\left.\exp\!\left(\,\dfrac{\lambda^{2}\cdot(b/n-a/n)^{2}}{8}\,\right)
		\right\slash \exp(\,\lambda\,\varepsilon\,)\,,
	\quad
	\textnormal{by Hoeffding's Lemma, Lemma \ref{lemma:HoeffdingLemma}}
\\
& = &
	\overset{n}{\underset{i=1}{\prod}}\;
	\left.\exp\!\left(\,\dfrac{\lambda^{2}\cdot(b-a)^{2}}{8\,n^{2}}\,\right)
		\right\slash \exp(\,\lambda\,\varepsilon\,)
\\
& = &
	\exp\!\left(\;
		n\cdot\dfrac{\lambda^{2}\cdot(b-a)^{2}}{8\,n^{2}}
		\; - \;
		\lambda\,\varepsilon
		\;\right)
\;\; = \;\;
	\exp\!\left(\;
		- \, \lambda\,\varepsilon
		\; + \;
		\dfrac{\lambda^{2}\cdot(b-a)^{2}}{8\,n}
		\;\right)
\end{eqnarray*}
Now, choosing \,$\lambda \,=\, \dfrac{4\,n\,\varepsilon}{(b-a)^{2}}$\, yields:
\begin{equation*}
- \, \lambda\,\varepsilon \; + \; \dfrac{\lambda^{2}\cdot(b-a)^{2}}{8\,n}
\;\; = \;\;
	- \, \dfrac{4\,n\,\varepsilon}{(b-a)^{2}}\cdot\varepsilon
	\; + \;
	\dfrac{16\,n^{2}\,\varepsilon^{2}}{}\cdot\dfrac{(b-a)^{2}}{8\,n}
\;\; = \;\;
	- \, \dfrac{4\,n\,\varepsilon^{2}}{(b-a)^{2}}
	\; + \; \dfrac{2\,n\,\varepsilon^{2}}{(b-a)^{2}}
\;\; = \;\;
	- \, \dfrac{2\,n\,\varepsilon^{2}}{(b-a)^{2}}
\end{equation*}
We therefore see that, via the above choice of \,$\lambda \,=\, \dfrac{4\,n\,\varepsilon}{(b-a)^{2}}$\,:
\begin{eqnarray*}
P\!\left(\;\,
	\dfrac{1}{n}\;\overset{n}{\underset{i=1}{\sum}}\,Z_{i} \, - \, \mu
	\,>\,\varepsilon
	\;\right)
& \leq &
	\exp\!\left(\;
		- \, \lambda\,\varepsilon
		\; + \;
		\dfrac{\lambda^{2}\cdot(b-a)^{2}}{8\,n}
		\;\right)
\;\; = \;\;
	\exp\!\left(\,
		-\,\dfrac{
			2\,n\,\varepsilon^{2}
			}{
			(b-a)^{2}
			}
		\;\right)
\end{eqnarray*}
On the other hand,
\begin{eqnarray*}
P\!\left(\;\,
	\dfrac{1}{n}\;\overset{n}{\underset{i=1}{\sum}}\,Z_{i} \, - \, \mu
	\,<\, -\,\varepsilon
	\;\right)
& \leq &
	P\!\left(\;\,
		\dfrac{1}{n}\;\overset{n}{\underset{i=1}{\sum}}\,Z_{i} \, - \, \mu
		\,\leq\,-\,\varepsilon
		\;\right)
\;\; = \;\;
	P\!\left(\;\,
		\dfrac{1}{n}\;\overset{n}{\underset{i=1}{\sum}}\,(Z_{i} \, - \, \mu)
		\,\leq\, -\,\varepsilon
		\;\right)
\\
& = &
	P\!\left(\;
		- \, \dfrac{1}{n}\;\overset{n}{\underset{i=1}{\sum}}\,(Z_{i} \, - \, \mu)
		\,\geq\, \varepsilon
		\;\right)
\,\;\; \leq \;\;\,
	\cdots
\,\;\; \leq \;\;\,
	\exp\!\left(\,
		-\,\dfrac{
			2\,n\,\varepsilon^{2}
			}{
			(b-a)^{2}
			}
		\;\right)
\end{eqnarray*}
Lastly, we may now conclude that, for each \,$\varepsilon > 0$\,,
\begin{eqnarray*}
P\!\left(\;\,\left\vert\,
	\dfrac{1}{n}\;\overset{n}{\underset{i=1}{\sum}}\,Z_{i} \, - \, \mu
	\,\right\vert
	\,>\,\varepsilon
	\;\right)
& = &
	P\!\left(\;
		\left\{\,
			\dfrac{1}{n}\;\overset{n}{\underset{i=1}{\sum}}\,Z_{i} \, - \, \mu
			\,>\,\varepsilon
			\,\right\}
		\;\bigcup\;
		\left\{\,
			\dfrac{1}{n}\;\overset{n}{\underset{i=1}{\sum}}\,Z_{i} \, - \, \mu
			\,<\,-\,\varepsilon
			\,\right\}
		\;\right)
\\
& \leq &
	P\!\left(\;
		\left\{\,
			\dfrac{1}{n}\;\overset{n}{\underset{i=1}{\sum}}\,Z_{i} \, - \, \mu
			\,>\,\varepsilon
			\,\right\}
		\;\right)
	\; + \;
	P\!\left(\;
		\left\{\,
			\dfrac{1}{n}\;\overset{n}{\underset{i=1}{\sum}}\,Z_{i} \, - \, \mu
			\,<\,-\,\varepsilon
			\,\right\}
		\;\right)
\\
& \leq &
	\exp\!\left(\;
		-\,\dfrac{
			2\,n\,\varepsilon^{2}
			}{
			(b-a)^{2}
			}
		\;\right)
	\; + \;
	\exp\!\left(\;
		-\,\dfrac{
			2\,n\,\varepsilon^{2}
			}{
			(b-a)^{2}
			}
		\;\right)
	\;\; = \;\;
	2\,\exp\!\left(\;
		-\,\dfrac{
			2\,n\,\varepsilon^{2}
			}{
			(b-a)^{2}
			}
		\;\right)\,.
\end{eqnarray*}
This completes the proof Hoeffding's Inequality.
\qed

          %%%%% ~~~~~~~~~~~~~~~~~~~~ %%%%%

\vskip 1.0cm
\begin{lemma}[Zero true risk implies empirical risk equals zero with probability one]
\mbox{}\vskip 0.1cm
\noindent
Suppose:
\begin{itemize}
\item
	$(\Omega,\mathcal{A},\mu)$ is a probability space, and
\item
	$(X,Y) : (\Omega,\mathcal{A},\mu) \longrightarrow \mathcal{X}\times\{0,1\}$\,
	is an $\left(\,\mathcal{X}\times\{0,1\}\right)$-valued random variable
	defined on $(\Omega,\mathcal{A},\mu)$.
\end{itemize}
If \,$h : \mathcal{X} \overset{{\color{white}-}}{\longrightarrow} \{0,1\}$\,
is a function with zero true risk, i.e.
\begin{equation*}
P\!\left(\,h(X) \overset{{\color{white}.}}{\neq} Y \,\right) \,=\, 0\,,
\end{equation*}
then its empirical risk equals zero with probability one, i.e.
\begin{equation*}
P_{\,D_{n}}\!\!\left(\;
	\textnormal{EmpiricalRisk}(\;h\,\overset{{\color{white}-}}{;}\,D_{n}\,) \,=\, 0
	\;\right)
\;\, = \;\;
P_{\,D_{n}}\!\!\left(\;
	\dfrac{1}{n}\;\overset{n}{\underset{i=1}{\sum}}\;
	I_{\left\{h(X_{i}) \,\overset{{\color{white}.}}{\neq}\, Y_{i}\right\}} \,=\, 0
	\;\right)
\;\, = \;\;
	1\,,
\quad
\textit{for each \,$n\in\N$}\,,
\end{equation*}
where
\,$D_{n} = \left((X_{1},Y_{1}),(X_{2},Y_{2}),\,\overset{{\color{white}\vert}}{\ldots}\,,(X_{n},Y_{n})\right)
	: \Omega \longrightarrow
	\left(\,\mathcal{X}\overset{{\color{white}.}}{\times}\{0,1\}\,\right)^{n}$\,
is such that the $(X_{i},Y_{i})$'s are I.I.D. copies of $(X,Y)$.
\end{lemma}
\proof
Note that
\begin{equation*}
P\!\left(\,h(X) \overset{{\color{white}.}}{\neq} Y \,\right) \,=\, 0
\quad\Longleftrightarrow\quad
P\!\left(\,h(X) \overset{{\color{white}-}}{=} Y \,\right) \,=\, 1
\end{equation*}
Hence,
\begin{eqnarray*}
P_{\,D_{n}}\!\!\left(\;
	\textnormal{EmpiricalRisk}(\;h\,\overset{{\color{white}-}}{;}\,D_{n}\,) \,=\, 0
	\;\right)
& = &
	P_{\,D_{n}}\!\!\left(\;
		\dfrac{1}{n}\;\overset{n}{\underset{i=1}{\sum}}\;
		I_{\left\{h(X_{i}) \,\overset{{\color{white}.}}{\neq}\, Y_{i}\right\}} \,=\, 0
		\;\right)
\\
& = &
	P_{\,D_{n}}\!\!\left(\;
		h(X_{i}) \,\overset{{\color{white}-}}{=}\, Y_{i}\,,\;\forall\;i\in\{1,\ldots,n\}
		\;\right)
\\
& = &
	\overset{n}{\underset{i=1}{\prod}}\;
	P\!\left(\; h(X_{i}) \,\overset{{\color{white}-}}{=}\, Y_{i} \;\right)
\\
& = &
	P\!\left(\; h(X) \,\overset{{\color{white}-}}{=}\, Y \;\right)^{n}
	\;\; = \;\;\, 1^{n} \;\; = \;\; 1\,,
\end{eqnarray*}
as required.
\qed

\begin{remark}
\mbox{}\vskip 0.1cm
\noindent
Note that the existence of a function \,$h : \mathcal{X} \longrightarrow \{0,1\}$\,
with zero risk, i.e.
\begin{equation*}
P\!\left(\,h(X) \overset{{\color{white}.}}{\neq} Y \,\right) \,=\, 0
\end{equation*}
implies that, with probability one, $Y$ is in fact a deterministic function of $X$.
\end{remark}

          %%%%% ~~~~~~~~~~~~~~~~~~~~ %%%%%

%\renewcommand{\theenumi}{\alph{enumi}}
%\renewcommand{\labelenumi}{\textnormal{(\theenumi)}$\;\;$}
\renewcommand{\theenumi}{\roman{enumi}}
\renewcommand{\labelenumi}{\textnormal{(\theenumi)}$\;\;$}

          %%%%% ~~~~~~~~~~~~~~~~~~~~ %%%%%
