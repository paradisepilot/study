
          %%%%% ~~~~~~~~~~~~~~~~~~~~ %%%%%

\section{La conjugaison des verbes r\'eguliers aux temps simples}
\setcounter{theorem}{0}
\setcounter{equation}{0}

\newcommand{\stemPresent}{parl}
\newcommand{\stemFutur}{parler}
\begin{center}
\textbf{Des verbes en -er (parler)}
\vskip 0.1cm
\begin{tabular}{|c||L{3cm}|L{3cm}|L{3cm}|L{3cm}|}
\hline
& Pr\'esent & Imparfait & Conditionnel & Futur \\
\hline
& \textit{do, am doing} & \textit{used to do, was doing} & \textit{would do} & \textit{will do} \\
\hline\hline
je            &	\stemPresent-e	&	\stemPresent-ais	&	\stemFutur-ais	 &	\stemFutur-ai	\\
tu            &	\stemPresent-es	&	\stemPresent-ais	&	\stemFutur-ais	&	\stemFutur-as	\\
il/elle/on  &	\stemPresent-e	&	\stemPresent-ait	&	\stemFutur-ait	&	\stemFutur-a	\\
nous       &	\stemPresent-ons	&	\stemPresent-ions	&	\stemFutur-ions	&	\stemFutur-ons	\\
vous       &	\stemPresent-ez	&	\stemPresent-iez	&	\stemFutur-iez		&	\stemFutur-ez	\\
ils/elles   &	\stemPresent-ent	&	\stemPresent-aient	&	\stemFutur-aient	&	\stemFutur-ont	\\
\hline
\end{tabular}
\end{center}

\begin{comment}
\renewcommand{\stemFutur}{choisir}
\renewcommand{\stemPresent}{chois}
\begin{center}
\textbf{Des verbes en -ir (choisir)}
\vskip 0.1cm
\begin{tabular}{|c||L{3cm}|L{3cm}|L{3cm}|L{3cm}|}
\hline
& Pr\'esent & Imparfait & Conditionnel & Futur \\
\hline
& \textit{do, am doing} & \textit{used to do, was doing} & \textit{would do} & \textit{will do} \\
\hline\hline
je           &	\stemPresent-is 	&	\stemPresent-iss-ais	&	\stemFutur-ais	 &	\stemFutur-ai	\\
tu           &	\stemPresent-is	&	\stemPresent-iss-ais	&	\stemFutur-ais	&	\stemFutur-as	\\
il/elle/on &	\stemPresent-it	&	\stemPresent-iss-ait	&	\stemFutur-ait	&	\stemFutur-a	\\
nous      &	\stemPresent-issons	&	\stemPresent-iss-ions	&	\stemFutur-ions	&	\stemFutur-ons	\\
vous      &	\stemPresent-issez	&	\stemPresent-iss-iez	&	\stemFutur-iez		&	\stemFutur-ez	\\
ils/elles  &	\stemPresent-issent	&	\stemPresent-iss-aient	&	\stemFutur-aient	&	\stemFutur-ont	\\
\hline
\end{tabular}
\end{center}
\end{comment}

\vskip 0.5cm
\renewcommand{\stemPresent}{fin}
\renewcommand{\stemFutur}{finir}
\begin{center}
\textbf{Des verbes en -ir (finir)}
\vskip 0.1cm
\begin{tabular}{|c||L{3cm}|L{3cm}|L{3cm}|L{3cm}|}
\hline
& Pr\'esent & Imparfait & Conditionnel & Futur \\
\hline
& \textit{do, am doing} & \textit{used to do, was doing} & \textit{would do} & \textit{will do} \\
\hline\hline
je           &	\stemPresent-is 	&	\stemPresent-iss-ais	&	\stemFutur-ais	 &	\stemFutur-ai	\\
tu           &	\stemPresent-is	&	\stemPresent-iss-ais	&	\stemFutur-ais	&	\stemFutur-as	\\
il/elle/on &	\stemPresent-it	&	\stemPresent-iss-ait	&	\stemFutur-ait	&	\stemFutur-a	\\
nous      &	\stemPresent-iss-ons	&	\stemPresent-iss-ions	&	\stemFutur-ions	&	\stemFutur-ons	\\
vous      &	\stemPresent-iss-ez	&	\stemPresent-iss-iez	&	\stemFutur-iez		&	\stemFutur-ez	\\
ils/elles  &	\stemPresent-iss-ent	&	\stemPresent-iss-aient	&	\stemFutur-aient	&	\stemFutur-ont	\\
\hline
\end{tabular}
\end{center}

\vskip 0.5cm
\renewcommand{\stemPresent}{r\'epond}
\renewcommand{\stemFutur}{r\'epondr}
\begin{center}
\textbf{Des verbes en -re (r\'epondre)}
\vskip 0.1cm
\begin{tabular}{|c||L{3cm}|L{3cm}|L{3cm}|L{3cm}|}
\hline
& Pr\'esent & Imparfait & Conditionnel & Futur \\
\hline
& \textit{do, am doing} & \textit{used to do, was doing} & \textit{would do} & \textit{will do} \\
\hline\hline
je           &	\stemPresent-s 	&	\stemPresent-ais	&	\stemFutur-ais	 &	\stemFutur-ai	\\
tu           &	\stemPresent-s	&	\stemPresent-ais	&	\stemFutur-ais	&	\stemFutur-as	\\
il/elle/on &	\stemPresent-	&	\stemPresent-ait	&	\stemFutur-ait	&	\stemFutur-a	\\
nous      &	\stemPresent-ons	&	\stemPresent-ions	&	\stemFutur-ions	&	\stemFutur-ons	\\
vous      &	\stemPresent-ez	&	\stemPresent-iez	&	\stemFutur-iez		&	\stemFutur-ez	\\
ils/elles  &	\stemPresent-ent	&	\stemPresent-aient	&	\stemFutur-aient	&	\stemFutur-ont	\\
\hline
\end{tabular}
\end{center}

\vskip 0.3cm
{\scriptsize
\noindent
\textbf{Tips:}
\begin{itemize}
\item	For each of Imparfait, Conditionnel, and Futur, the suffixes are the same across all three verb groups.
\item	For Conditionnel and Futur, conjugation follows this formula: infinitive + suffix (except the final ``e" is dropped for -re verbs).
\item	For {\color{red}I}mparfait and Cond{\color{red}i}tionnel, all suffixes contain an the letter {\color{red}i}. 
\end{itemize}
}
          %%%%% ~~~~~~~~~~~~~~~~~~~~ %%%%%

\clearpage
\section{La conjugaison aux temps compos\'es}
\setcounter{theorem}{0}
\setcounter{equation}{0}

\begin{flushleft}
\textbf{Avoir}
\vskip 0.1cm
\begin{tabular}{|c||L{2.5cm}|L{2.5cm}|L{2.5cm}|L{2.5cm}|}
\hline
& Pr\'esent & Imparfait & Conditionnel & Futur \\
\hline
& \textit{have, am having} & \textit{used to have, was having} & \textit{would have} & \textit{will have} \\
\hline\hline
je            &	ai	&	avais	&	aurais	 &	aurai	\\
tu            &	as 	&	avais	&	aurais	&	auras	\\
il/elle/on  &	a	&	avait	&	aurait	&	aura	\\
nous       &	avons	&	avions	&	aurions	&	aurons	\\
vous       &	avez	&	aviez	&	auriez		&	aurez	\\
ils/elles   &	ont 	&	avaient	&	auraient	&	auront	\\
\hline
\end{tabular}
\end{flushleft}

\begin{flushleft}
\textbf{Parler}
\vskip 0.1cm
\begin{tabular}{|c||L{2.5cm}|L{2.5cm}|L{2.5cm}|L{2.5cm}|}
\hline
& Pass\'e compos\'e & Plus-que-parfait & Conditionnel Pass\'e & Futur ant\'erieur \\
\hline
& \textit{did, have done} & \textit{had done} & \textit{would have done} & \textit{will have done} \\
\hline\hline
je            &	ai	&	avais	&	aurais	 &	aurai	\\
tu            &	as 	&	avais	&	aurais	&	auras	\\
il/elle/on  &	a	&	avait	&	aurait	&	aura	\\
nous       &	avons	&	avions	&	aurions	&	aurons	\\
vous       &	avez	&	aviez	&	auriez		&	aurez	\\
ils/elles   &	ont 	&	avaient	&	auraient	&	auront	\\
\hline
\end{tabular}
\quad$+$\quad parl\'e
\end{flushleft}

\clearpage
%\vskip 0.5cm
\begin{flushleft}
\textbf{\^Etre}
\vskip 0.1cm
\begin{tabular}{|c||L{2.5cm}|L{2.5cm}|L{2.5cm}|L{2.5cm}|}
\hline
& Pr\'esent & Imparfait & Conditionnel & Futur \\
\hline
& \textit{am, am being} & \textit{used to be, was being} & \textit{would be} & \textit{will be} \\
\hline\hline
je            &	suis	&	\'etais	&	serais	 &	serai	\\
tu            &	es 	&	\'etais	&	serais	&	seras	\\
il/elle/on  &	est	&	\'etait	&	serait	&	sera	\\
nous       &	sommes	&	\'etions	&	serions	&	serons	\\
vous       &	\^etes	&	\'etiez	&	seriez		&	serez	\\
ils/elles   &	sont 	&	\'etaient	&	seraient	&	seront	\\
\hline
\end{tabular}
\end{flushleft}

\begin{flushleft}
\textbf{Aller, Tomber}
\vskip 0.1cm
\begin{tabular}{|c||L{2.5cm}|L{2.5cm}|L{2.5cm}|L{2.5cm}|}
\hline
& Pass\'e compos\'e & Plus-que-parfait & Conditionnel Pass\'e & Futur ant\'erieur \\
\hline
& \textit{did, have done} & \textit{had done} & \textit{would have done} & \textit{will have done} \\
\hline\hline
je            &	suis	&	\'etais	&	serais	 &	serai	\\
tu            &	es 	&	\'etais	&	serais	&	seras	\\
il/elle/on  &	est	&	\'etait	&	serait	&	sera	\\
nous       &	sommes	&	\'etions	&	serions	&	serons	\\
vous       &	\^etes	&	\'etiez	&	seriez		&	serez	\\
ils/elles   &	sont 	&	\'etaient	&	seraient	&	seront	\\
\hline
\end{tabular}
\;$+$
$\begin{array}{c}
\textnormal{all\'e}
\\
\textnormal{tomb\'e}
\end{array}$
$+$ (e)(s)
\end{flushleft}

          %%%%% ~~~~~~~~~~~~~~~~~~~~ %%%%%

\clearpage
\section{La conjugaison des verbes irr\'eguliers courant aux temps simples}
\setcounter{theorem}{0}
\setcounter{equation}{0}

\renewcommand{\stemPresent}{cour}
\renewcommand{\stemFutur}{courr}
\begin{center}
\textbf{courir}
\vskip 0.1cm
\begin{tabular}{|c||L{3cm}|L{3cm}|L{3cm}|L{3cm}|}
\hline
& Pr\'esent & Imparfait & Conditionnel & Futur \\
\hline\hline
je           &	\stemPresent-s 	&	\stemPresent-ais	&	\stemFutur-ais	 &	\stemFutur-ai	\\
tu           &	\stemPresent-s	&	\stemPresent-ais	&	\stemFutur-ais	&	\stemFutur-as	\\
il/elle/on &	\stemPresent-t	&	\stemPresent-ait	&	\stemFutur-ait	&	\stemFutur-a	\\
nous      &	\stemPresent-ons	&	\stemPresent-ions	&	\stemFutur-ions	&	\stemFutur-ons	\\
vous      &	\stemPresent-ez	&	\stemPresent-iez	&	\stemFutur-iez		&	\stemFutur-ez	\\
ils/elles  &	\stemPresent-ent	&	\stemPresent-aient	&	\stemFutur-aient	&	\stemFutur-ont	\\
\hline
\end{tabular}
\end{center}







